\input tex/epsf.tex
\font\sixteen=cmbx16
\font\twelve=cmr12
\font\fonteautor=cmbx12
\font\fonteemail=cmtt10
\font\twelvenegit=cmbxti12
\font\twelvebold=cmbx12
\font\trezebold=cmbx13
\font\twelveit=cmsl12
\font\monodoze=cmtt12
\font\it=cmti12
\voffset=0,959994cm % 3,5cm de margem superior e 2,5cm inferior
\parskip=6pt

\def\titulo#1{{\noindent\sixteen\hbox to\hsize{\hfill#1\hfill}}}
\def\autor#1{{\noindent\fonteautor\hbox to\hsize{\hfill#1\hfill}}}
\def\email#1{{\noindent\fonteemail\hbox to\hsize{\hfill#1\hfill}}}
\def\negrito#1{{\twelvebold#1}}
\def\italico#1{{\twelveit#1}}
\def\monoespaco#1{{\monodoze#1}}
\def\iniciocodigo{\lineskip=0pt\parskip=0pt}
\def\fimcodigo{\twelve\parskip=0pt plus 1pt\lineskip=1pt}

\long\def\abstract#1{\parshape 10 0.8cm 13.4cm 0.8cm 13.4cm
0.8cm 13.4cm 0.8cm 13.4cm 0.8cm 13.4cm 0.8cm 13.4cm 0.8cm 13.4cm
0.8cm 13.4cm 0.8cm 13.4cm 0.8cm 13.4cm
\noindent{{\twelvenegit Abstract: }\twelveit #1}}

\def\resumo#1{\parshape  10 0.8cm 13.4cm 0.8cm 13.4cm
0.8cm 13.4cm 0.8cm 13.4cm 0.8cm 13.4cm 0.8cm 13.4cm 0.8cm 13.4cm
0.8cm 13.4cm 0.8cm 13.4cm 0.8cm 13.4cm
\noindent{{\twelvenegit Resumo: }\twelveit #1}}

\def\secao#1{\vskip12pt\noindent{\trezebold#1}\parshape 1 0cm 15cm}
\def\subsecao#1{\vskip12pt\noindent{\twelvebold#1}}
\def\referencia#1{\vskip6pt\parshape 5 0cm 15cm 0.5cm 14.5cm 0.5cm 14.5cm
0.5cm 14.5cm 0.5cm 14.5cm {\twelve\noindent#1}}

%@* .

\twelve
\vskip12pt
\titulo{Windowing Interface Weaver}
\vskip12pt
\autor{Thiago Leucz Astrizi}
\vskip6pt
\email{thiago@@bitbitbit.com.br}
\vskip6pt

\abstract{This article contains the implementation of a portable
  Windowing API, which can be used to create a single window in
  Windows, Linux, BSD, or a canvas in Web Assembly running in a web
  browser. You can set a fullscreen mode, change the resolution in the
  window, resize it, use OpenGL commands and get input from mouse and
  keyboard. All this can be achieved by the portable API defined in
  this work.}

\secao{1. Introduction}

A graphical computer program needs a space where it can draw in the
screen. In some environments, lke video-games, each program always
controll the entire screen automatically, without asking for it to
some operating system. But when a computer program runs in a computer
with some modern graphical environment, usually it is necessary to ask
for the creation of some region called ``window''. There the program
have control over the content and can draw freely.

Besides creating the window, it is also important to have control if
we are in full-screen or not if we are in an environment that allows
this. And also change the size of our window in pixels. If we are in
fullscreen mode, this means changing the resolution of the
screen. Otherwise, this means changing the window size. This is
important because depending on the used visual effects, we could need
to draw less pixels in the screen for performance reasons.

\subsecao{1.1. Literary Programming and Notation}

This article uses the technique of ``Literary Programming'' to develop
our random number generator API. This technique was presented at
[Knuth, 1984] and have as objective develop software in a way that a
computer program to be compiled is exactly equal a document written
for human beings detailing and explaining the code. This document is
not independent of the source code, it is the project source
code. Automated tools are used to extract the code from this document,
sort it in the right order and produce the code that is passed to a
compiler.

For example, in this article we will define two different
files: \monoespaco{window.c} and \monoespaco{window.h}. Both of them
can be inserted statically in any project or compiled as a shared
library. The content of \monoespaco{window.h} is:

\iniciocodigo
@(src/window.h@>=
#ifndef WEAVER_WINDOW
#define WEAVER_WINDOW
#ifdef __cplusplus
extern "C" {
#endif
@<Macro Definition@>
@<Window Declarations@>
#ifdef __cplusplus
}
#endif
#endif
@
\fimcodigo

The two first lines and the last one are macros that ensure that the
function declaration and variables from this file will always be
included at most once in a compiling unit. We also put macros to check
if we are compiling this as C or C++. If we are in C++, we assure the
compiler that all our functions will be in C-style. We never will
modify them with operator overload, for example. This makes the code
became more compact.

The red parts in the above code shows that some code will be inserted
there in the future. In ``RNG Declarations'', for example, we will put
there function declarations.

Each piece of code have a title, that in the above example is
\monoespaco{src/window.h}. The title shows where the code will be placed.
In the case above, the code will directly to a file. In future pieces
of code, we will have different titles, including titles matching
exactly the names present in the red code above. If a piece of code
have as title a name matching the red part in some code, then that
code is positioned exactly in that red part when compiling the code.

\subsecao{1.2. API functions that will be defined}

In this article we will define the following functions:

\iniciocodigo
@<Window Declarations@>=
void _Wcreate_window(void);
@
\fimcodigo

This function will create a new window. By default, it will create a
fullscreen window. If the macro \monoespaco{W\_WINDOW\_NO\_FULLSCREEN}
is defined, instead it will create a non-fullscreen window.

If the macro \monoespaco{W\_DEBUG\_WINDOW} is defined, this function
also will print in the screen information about the graphical
environment. Its resolution for example. But possibly other
information that can be relevant.

\iniciocodigo
@<Window Declarations@>=
void _Wdestroy_window(void);
@
\fimcodigo

This function closes the window, freeing any resource allocated during
the window creation by the previous function. This function must be
invoked only after a window was created.

\secao{2. Creating and Managing the Window}

\subsecao{2.1. Creating a Window in X}

The X Window System, also known as X11, is a windowing system present
in many Operating System, as Linux, BSD and even in MacOX, where it
exists to ensure compatibility with older programs developed before
their current windowing system. We will begin our window creation with
X11 because is the most widely present windowing system.

X11 works in a client-server architecture. When we create a graphical
program, we create a client that communicates with X server using
sockets. All operations like window creation, resizing windows and
more are made when with the client asking for them sending requests to
the server, which executes the requests if possible.

We expect to use X11 when we are not compiling programs to Windows or
to Web Assembly, as both the Windows Operating System and web browsers
do not have a X11 server. If we will use X11, we need the following
header:

\iniciocodigo
@<Headers@>=
#if !defined(_WIN32) && !defined(__EMSCRIPTEN__)
#include <X11/Xlib.h>
#endif
@
\fimcodigo

In X11, as we need to communicate with a server, we create a window
following the steps below:

1. We open a connection with the server. If we succeed, we get as
response a list of relevant information about the screen.

2. We read from the response which is the default screen where we
should create our window (a computer could have many different
monitors and screens).

3. We also read the screen resolution.

4. We send a new message to the server asking to create a new
window. We will do it in the default screen, The window sizecan be
controlled by macros \monoespaco{W\_WINDOW\_DEFAULT\_RESOLUTION_X} and
\monoespaco{W\_WINDOW\_DEFAULT\_RESOLUTION_Y}. If one of these macros
have a non-positive value, we interpret this as a instruction to
create the largest possible window that still fits in the screen.

The above steps are implemented using the functions and macros below:

\iniciocodigo
@<X11: Create Window@>=
#if !defined(_WIN32) && !defined(__EMSCRIPTEN__)
/* Step 1: */
display = XOpenDisplay(NULL);
if(display == NULL){
  fprintf(stderr, "ERROR: Could not connect to a X Server.\n");
  exit(1);
}
/* Step 2: */
screen = DefaultScreen(display);
/* Step 3: */
screen_resolution_x = DisplayWidth(display, screen);
screen_resolution_y = DisplayHeight(display, screen);
/* Step 4: */
window = XCreateSimpleWindow(display, // X11 connection
                             DefaultRootWindow(display),// Parent window
                             0, 0, // Window position
                             (W_WINDOW_DEFAULT_RESOLUTION_X <= 0)?
                               (screen_resolution_x):
                               (W_WINDOW_DEFAULT_RESOLUTION_X), // Width
                             (W_WINDOW_DEFAULT_RESOLUTION_Y <= 0)?
                               (screen_resolution_y):
                               (W_WINDOW_DEFAULT_RESOLUTION_Y), // Height
                             0, 0, // Border width and color
                             0); // Default window color
#endif
@
\fimcodigo

This code assumes that we have the following declared variables:

\iniciocodigo
@<Local Variables@>=
static int screen_resolution_x, screen_resolution_y; //Screen resolution
#if !defined(_WIN32) && !defined(__EMSCRIPTEN__)
static Display *display; // Connection with server and screen info
static int screen;       // Deafult screen id
static Window window;    // Created window struct
#endif
@
\fimcodigo

And also assumes that we included the header to print messages in the
standard error output and to exit the procram in case of error:

\iniciocodigo
@<Headers@>+=
#include <stdio.h>
#include <stdlib.h>
@
\fimcodigo

Now we need to consider the scenario where the macros that determines
the default window resolution are not defined. In this case we define
it to have the standard value zero:

\iniciocodigo
@<Macro Definition@>=
#if !defined(W_WINDOW_DEFAULT_RESOLUTION_X)
#define W_WINDOW_DEFAULT_RESOLUTION_X 0
#endif
#if !defined(W_WINDOW_DEFAULT_RESOLUTION_Y)
#define W_WINDOW_DEFAULT_RESOLUTION_Y 0
#endif
@
\fimcodigo

The code defined above creates a window, but it does not mean that the
created window is drawn in the screen. Before drawing the window in
the screen we can adjust some of its properties.

One of the relevant information is what kind of events is relevant to
pass to the program when something happen to the window. For example,
we do not consider relevant if the user interacts with another window
making our window lose focus. But other events, like the information
that the user pressed a key is relevant and our program should be
informed.

The list of events that we consider relevant is: the window is created
or destroyed, the user presses or releases a mouse button, presses or
releases a key in keyboard and if the user moves the mouse pointer. If
we do not ask the server to send this information, our program would
not be able to know if the user interacts with it pressing keys.

\iniciocodigo
@<X11: Criar Create Window@>+=
#if !defined(_WIN32) && !defined(__EMSCRIPTEN__)
XSelectInput(display, window, StructureNotifyMask | KeyPressMask |
                              KeyReleaseMask | ButtonPressMask |
                              ButtonReleaseMask | PointerMotionMask);
#endif
@
\fimcodigo

Another important thing is choose the name for our window. Usually
this name is presented by the window manager. We let the user choose
the name setting the macro \monoespaco{W\_WINDOW\_NAME}:

\iniciocodigo
@<X11: Create Window@>+=
#if !defined(_WIN32) && !defined(__EMSCRIPTEN__)
XStoreName(display, window, W_WINDOW_NAME);
#endif
@
\fimcodigo

If this macro is not defined, we use an empty string:

@<Macro Definition@>+=
#if !defined(W_WINDOW_NAME)
#define W_WINDOW_NAME ""
#endif
@
\fimcodigo

After adjusting all the configurations, we can draw the created
window. For this, we send a request to the server X and wait in a loop
until we are notified that the window was created (we asked to be
notified of this event). The code for this is:

\iniciocodigo
@<X11: Create Window@>+=
#if !defined(_WIN32) && !defined(__EMSCRIPTEN__)
XMapWindow(display, window);
{
  XEvent e;
  do{
    XNextEvent(display, &e);
  } while(e.type != MapNotify);
}
#endif
@
\fimcodigo


\subsecao{2.X. Structure of Window Creation Function}

Our function that creates a new window will execute the adequate code,
depending on the graphical environment where it is being executed:

\iniciocodigo
@<Functions@>=
void _Wcreate_window(void){
  @<X11: Create Window@>
}
@
\fimcodigo

\subsecao{2.X+1. Closing a Window in X}

Closing a window in X11 means calling the function that asks the
server to close the window and also closing the connection with the
server. This is done calling the two functions below:

\iniciocodigo
@<Functions@>=
void _Wdestroy_window(void){
  XDestroyWindow(display, window);
  XCloseDisplay(display);
}
@
\fimcodigo


\subsecao{X. Final File Sctructure}

The file with all the necessary source code for the functions defined
in this article will have the following organization:

\iniciocodigo
@(src/window.c)@>=
#include "window.h"
@<Headers@>
@<Local Variables@>
@<Functions@>
@
\fimcodigo


\fim
