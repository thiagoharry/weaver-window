\font\sixteen=cmbx16
\font\twelve=cmr12
\font\fonteautor=cmbx12
\font\fonteemail=cmtt10
\font\twelvenegit=cmbxti12
\font\twelvebold=cmbx12
\font\trezebold=cmbx13
\font\twelveit=cmsl12
\font\monodoze=cmtt12
\font\it=cmti12
\voffset=0,959994cm % 3,5cm de margem superior e 2,5cm inferior
\parskip=6pt

\def\titulo#1{{\noindent\sixteen\hbox to\hsize{\hfill#1\hfill}}}
\def\autor#1{{\noindent\fonteautor\hbox to\hsize{\hfill#1\hfill}}}
\def\email#1{{\noindent\fonteemail\hbox to\hsize{\hfill#1\hfill}}}
\def\negrito#1{{\twelvebold#1}}
\def\italico#1{{\twelveit#1}}
\def\monoespaco#1{{\monodoze#1}}
\def\iniciocodigo{\lineskip=0pt\parskip=0pt}
\def\fimcodigo{\twelve\parskip=0pt plus 1pt\lineskip=1pt}

\long\def\abstract#1{\parshape 10 0.8cm 13.4cm 0.8cm 13.4cm
0.8cm 13.4cm 0.8cm 13.4cm 0.8cm 13.4cm 0.8cm 13.4cm 0.8cm 13.4cm
0.8cm 13.4cm 0.8cm 13.4cm 0.8cm 13.4cm
\noindent{{\twelvenegit Abstract: }\twelveit #1}}

\def\resumo#1{\parshape  10 0.8cm 13.4cm 0.8cm 13.4cm
0.8cm 13.4cm 0.8cm 13.4cm 0.8cm 13.4cm 0.8cm 13.4cm 0.8cm 13.4cm
0.8cm 13.4cm 0.8cm 13.4cm 0.8cm 13.4cm
\noindent{{\twelvenegit Resumo: }\twelveit #1}}

\def\secao#1{\vskip12pt\noindent{\trezebold#1}\parshape 1 0cm 15cm}
\def\subsecao#1{\vskip12pt\noindent{\twelvebold#1}}
\def\subsubsecao#1{\vskip12pt\noindent{\negrito{#1}}}
\def\referencia#1{\vskip6pt\parshape 5 0cm 15cm 0.5cm 14.5cm 0.5cm 14.5cm
0.5cm 14.5cm 0.5cm 14.5cm {\twelve\noindent#1}}

%@* .

\twelve
\vskip12pt
\titulo{Interface de Janela Weaver}
\vskip12pt
\autor{Thiago Leucz Astrizi}
\vskip6pt
\email{thiago@@bitbitbit.com.br}
\vskip6pt

\abstract{This article contains the implementation of a portable
  Windowing API, which can be used to create a single window in
  Windows, Linux, BSD, or a canvas in Web Assembly running in a web
  browser. You can set a fullscreen mode, change to windowed mode,
  resize it, use OpenGL commands and get input from mouse and
  keyboard. All this can be achieved by the portable API defined in
  this work.}


\vskip 0.5cm plus 3pt minus 3pt

\resumo{Este artigo contém a implementação de uma interface de uso de
  janelas. Ela pode ser usada para criar uma única janela no Windows,
  Linux, BSD, ou então um ``canvas'' rodando Web Assembly em um
  navegador de Internet. Você pode ativar um modo de tela cheia, mudar
  para um modo em janela, mudar o tamanho da janela, usar comandos
  OpenGL e obter entrada do mouse e teclado. Tudo isso pode ser obtido
  pela interface de aplicação portável definida neste trabalho.}

\secao{1. Introdução}

Um programa de computador gráfico precisa de um espaço no qual ele
pode desenhar na tela. Em alguns ambientes, como videogames, por
exemplo, cada programa em execução simplesmente tem controle de toda a
tela automaticamente sem que seja necessário reservá-lo ou pedi-lo
para um Sistema Operacional. Por outro lado, quando um programa
executa em um computador com algum ambiente gráfico moderno, é
necessário pedir para que uma região chamada ``janela'' seja
criada. Nela o programa passa a ter controle sobre o seu conteúdo e
pode desenhar na região.

Além de criar uma janela, é importante que tenhamos a capacidade de
entrar e sair do modo tela-cheia se estivermos em ambiente que permite
isso. E se estivermos em modo de janela, mudar o tamanho da
janela.

\subsecao{1.1. Programação Literária e Notação}

Este artigo utiliza a técnica de ``Programação Literária'' para
desenvolver a API de gerador de números aleatórios. Esta técnica foi
apresentada em [Knuth, 1984] e tem por objetivo desenvolver
\italico{softwares} de tal forma que um programa de computador a ser compilado
é exatamente igual a um documento escrito para pessoas detalhando e
explicando o código. O presente documento não é algo independente do
código, mas sim consiste no próprio código-fonte do projeto.
Ferramentas automáticas são utilizadas para extrair o código deste
documento, colocá-lo na ordem correta e produzir o código que é
passado para o compilador.

Por exemplo, neste artigo serão definidos dois arquivos
diferentes: \monoespaco{window.c} e \monoespaco{window.h}, os quais
podem ser inseridos estaticamente em qualquer projeto, ou compilados
como uma biblioteca compartilhada. O conteúdo de \monoespaco{window.h}
é:

\iniciocodigo
@(src/window.h@>=
#ifndef WEAVER_WINDOW
#define WEAVER_WINDOW
#ifdef __cplusplus
extern "C" {
#endif
#include <stdbool.h> // Define tipo 'bool'
@<Cabeçalho OpenGL@>
@<Define Macros@>
@<Declarações de Janela@>
#ifdef __cplusplus
}
#endif
#endif
@
\fimcodigo

As duas primeiras linhas assim como a última são macros que impedem
que garantem que as funções e variáveis declaradas ali serão inseridas
no máximo uma só vez em cada unidade de compilação. Também colocamos
macros para checar se estamos compilando o código como C ou C++. Se
estivermos em C++, avisamos o compilador que estamos definindo tudo
como código C e garantimos que não vamos modificar nada usando
sobrecarga de operadores. O código poderá assim ser armazenado de
maneira mais compacta.

As partes vermelhas no código acima mostram que código será inserido
ali no futuro.

Cada trecho de código tem um título, que no caso acima
é \monoespaco{src/window.h}. O título indica onde o código será
inserido. No caso acima, o código irá para um arquivo. Em trechos de
código futuros, haverá diversos títulos, inclusive um com exatamente o
nome das partes em vermelho do código acima. Se o título de um trecho
de código é igual um trecho em vermelho a ser inserido, é ali que tal
trecho de código deve ser posicionado no processo de compilação.

Como um segundo exemplo de código, também declararemos aqui que quando
estamos em modo de depuração (ou seja, quando a
macro \monoespaco{W\_DEBUG\_WINDOW} está definida) iremos precisar das
declarações de funções de entrada e saída padrão, já que nosso código
se tornará mais verboso:

\iniciocodigo
@<Cabeçalhos@>=
#if defined(W_DEBUG_WINDOW)
#include <stdio.h>
#endif
@
\fimcodigo

Note que ainda não informamos onde exatamente o código denominado
``Cabeçalho'' será colocado. Faremos isso posteriormente.

\subsecao{1.2. Funções de API a serem Definidas}

Neste artigo iremos definir as seguintes funções:

\iniciocodigo
@<Declarações de Janela@>=
bool _Wcreate_window(void);
@
\fimcodigo

Esta é a função que irá criar uma nova janela. Por padrão, uma janela
em tela cheia. Se a macro \monoespaco{W\_WINDOW\_NO\_FULLSCREEN}
estiver definida, ao invés disso, ele criará uma janela que não está
em modo de tela-cheia (se suportado pelo sistema).

Se a macro \monoespaco{W\_DEBUG\_WINDOW} estiver definida, esta função
também imprimirá na tela informação sobre o ambiente gráfico. Sua
resolução, por exemplo, ou possivelmente outras informações que possam
ser relevantes.

Em caso de erro, a função retornará falso.

\iniciocodigo
@<Declarações de Janela@>=
bool _Wdestroy_window(void);
@
\fimcodigo

Esta função vai fechar a janela aberta, liberando qualquer recurso que
tenha sido alocado pela função anterior ao criar janela. A função deve
ser invocada sempre após a janela já ter sido criada. Em caso de erro,
retorna falso.

\iniciocodigo
@<Declarações de Janela@>=
bool _Wrender_window(void);
@
\fimcodigo

Esta função irá efetivamente renderizar na tela todos os comandos
OpenGL pendentes desde a última renderização. Retornará falso em caso
de erro. Esta função provavelmente será chamada no fim de cada
iteração de um laço principal.

\iniciocodigo
@<Declarações de Janela@>=
bool _Wget_screen_resolution(int *resolution_x, int *resolution_y);
@
\fimcodigo

Esta função obtém a resolução da tela e a armazena nos ponteiros
passados. Se houver mais de uma, retornará aquela que o sistema
identifica como sendo a principal. Se um erro ocorrer, ela retornará
falso e a resolução será armazenada como zero.

\iniciocodigo
@<Declarações de Janela@>=
bool _Wget_window_size(int *width, int *height);
@
\fimcodigo

Esta função obtém o tamanho em pixels da janela atual e armazena a
informação nos ponteiros passados como argumento. Se nós não temos
nenhuma janela, ou em caso de erro, a função retorna falso e armazena
zero nos ponteiros. Já em caso de sucesso, ela retorna verdadeiro e
armazena o resultado correto nos ponteiros.

\iniciocodigo
@<Declarações de Janela@>=
void _Wget_window_input(unsigned long current_time);
@
\fimcodigo

Esta função periodicamente deve ser chamada para atualizar o estado do
teclado e mouse. O primeiro argumento é o tempo atual medido em alguma
unidade de tempo. Os próximos argumentos são estruturas que
representam teclado e mouse a serem atualizadas.

\iniciocodigo
@<Declarações de Janela@>=
void _Wflush_window_input(void);
@
\fimcodigo

Esta função limpa o estado de nosso teclado e mouse. Ela deve ser
chamada toda vez que pararmos de ler periodicamente o mouse e teclado
com chamadas a \monoespaco{\_Wget\_window\_input}. O resultado de
chamar esta função é que reiniciamos o estado de nosso mouse e
teclado, parando de considerar qualquer tecla como pressionada ou
solta.

\secao{2. Criando e Gerenciando a Janela}

Nesta seção definiremos funções e códigos que envolvem criar uma
janela e criar um contexto OpenGL compatível com OpenGL ES 2.0 nelas.

\subsecao{2.1. Obtendo a Resolução da Tela}

Antes de criar nossa janela, é importante checar qual a resolução da
tela e da janela. Se estivermos iniciando em modo de tela cheia, este
será o tamanho da nossa janela. Isso é feito de maneira diferente
dependendo do sistema operacional.

Vamos também criar uma variável estática que funcionará como um cache
que memorizará o último tamanho da janela em pixels que
obtemos. Memorizar o tamanho da janela é importante porque precisamos
deste valor para depois transformar as coordenadas da posição do mouse
que iremos medir. Praticamente todos os ambientes informam as
coordenadas na janela usando como origem o canto superior
esquerdo. Contudo, a API Weaver prefere usar como origem o canto
inferior esquerdo para estar mais de acordo com a convenção
matemática. Para fazer a transformação de coordenada, é importante
memorizar o tamanho da janela para não termos que medi-la toda hora:

\iniciocodigo
@<Variáveis Locais@>=
static int window_size_y = 0;
@
\fimcodigo

\subsubsecao{2.1.1. Obtendo a Resolução da Tela no X}

O Sistema de Janelas X, também conhecido como X11, é um sistema de
janelas presente em muitos Sistemas Operacionais, como Linux, BSD, e
até mesmo no MacOX X, onde ele está presente para garantir
compatibilidade com programas mais antigos desenvolvidos antes de seu
sistema de janelas atual. Iremos começar com a implementação de nossa
função no X11 por ser o sistema de janelas mais amplamente presente.

O X11 funciona em uma arquitetura de cliente-servidor. Quando criamos
um programa gráfico, criamos um cliente que se comunica com o servidor
X usando \italico{sockets}. Todas as operações como a criação de
janelas, redimencionar a janela e mais, são feitas com o cliente
pedindo para que elas sejam feitas para o servidor, que executa os
pedidos se possível.

Iremos usar o X11 sempre que não estivermos usando o Windows (que não
o fornece) e nem estivermos compilando Web Assembly (navegadores de
Internet também não o implementam). Antes de usar o X11, precisamos
inserir seu cabeçalho relevante:

\iniciocodigo
@<Cabeçalhos@>=
#if !defined(_WIN32) && !defined(__EMSCRIPTEN__)
#include <X11/Xlib.h>
#endif
@
\fimcodigo

No X11, como temos que nos comunicar com um servidor, assumimos que
temos uma variável com algum tipo de estrutura de dados com informação
sobre a nossa conexão. Esta variável é chamada de
``\monoespaco{display}''.

Usando tal conexão, que assumimos ser uma variável estática
em \monoespaco{window.c}, ler a resolução da tela é feito com a função
abaixo:

\iniciocodigo
@<Funções da API@>=
#if !defined(_WIN32) && !defined(__EMSCRIPTEN__)
bool _Wget_screen_resolution(int *resolution_x, int *resolution_y){
  bool keep_alive = true; //Devemos manter a conexão ativa?
  // A primeira coisa a ser feita é chamar esta função para preparar o X11
  // para código multi-thread:
  XInitThreads();
  // Se não temos uma conexcão ativa, criamos ela:
  if(display == NULL){
    display = XOpenDisplay(NULL);
    keep_alive = false;
  }
  // Obtemos a tela padrão:
  int screen = DefaultScreen(display);
  // E perguntamos a resolução
  *resolution_x = DisplayWidth(display, screen);
  *resolution_y = DisplayHeight(display, screen);
  // Se não havia uma conexão ativa ao servidor X antes de invocar a função,
  // fechamos a conexão que abrimos para obter a resolução:
  if(!keep_alive){
    XCloseDisplay(display);
    display = NULL;
  }
  // Se tudo deu certo podemos retornar verdadeiro:
  return true;
}
#endif
@
\fimcodigo

Claro, temos que assumir a existência da seguinte variável estática:

\iniciocodigo
@<Variáveis Locais@>=
#if !defined(_WIN32) && !defined(__EMSCRIPTEN__)
static Display *display = NULL; //Conexão com servidor e info sobre tela
#endif
@
\fimcodigo

\subsubsecao{2.1.2. Obtendo a Resolução da Tela no Web Assembly}

Se estivermos executando em Web Assembly, então assumimos estar em um
Navegador de Internet. A nossa ``janela'' será um ``canvas'' HTML. E
nesse caso para obter o tamanho da tela, temos que recorrer à
invocação de Javascript. Fazemos isso com ajuda da API do Emscripten:

\iniciocodigo
@<Funções da API@>+=
#if defined(__EMSCRIPTEN__)
bool _Wget_screen_resolution(int *resolution_x, int *resolution_y){
  *resolution_x = EM_ASM_INT({
    return window.screen.width * window.devicePixelRatio;
  });
  *resolution_y = EM_ASM_INT({
    return window.screen.height * window.devicePixelRatio;
  });
  return true;
}
#endif
@
\fimcodigo


\subsubsecao{2.1.3. Obtendo a Resolução da Tela no Windows}

Finalmente, no Windows, isso é feito simplesmente chamando uma função
da API que nos dá esta informação:

\iniciocodigo
@<Funções da API@>+=
#if defined(_WIN32)
bool _Wget_screen_resolution(int *resolution_x, int *resolution_y){
  *resolution_x = GetSystemMetrics(SM_CXSCREEN);
  *resolution_y = GetSystemMetrics(SM_CYSCREEN);
  return true;
}
#endif
@
\fimcodigo

\subsecao{2.2. Criando uma Janela}

Agora descreveremos como criar uma janela nos diferentes tipos de
ambiente que suportamos: ambiente gráfico X11, Windows e rodando em
navegador web com Web Assembly.

\subsubsecao{2.2.1. Criando uma Janela no X}

Os passos para criar uma janela no X11 são:

0. Avisamos a biblioteca X que pode ser que múltiplas threads tentem
se comunicar com ele. É raro haver motivos para se fazer isso, mas é
útil se preparar só para o caso de acontecer. Essa precisa ser a
primeira coisa a ser feita antes de usar outras funções da biblioteca.

1. Abrimos uma conexão com o servidor. Se isso for bem-sucedido, o
servidor nos revela várias informações relevantes sobre a tela.

2. Lemos a resolução da tela com a função que definimos na seção 2.1.

3. Enviamos uma nova mensagem para o servidor pedindo que a janela
seja criada. Em princípio criaremos uma janela com o máximo de tamanho
permitido dada a resolução da tela.

4. Memorizamos o valor inicial da altura da janela.

Estes passos são implementados por meio das seguintes funções e
macros:

\iniciocodigo
@<X11: Criar Janela@>=
#if !defined(_WIN32) && !defined(__EMSCRIPTEN__)
int screen_resolution_x, screen_resolution_y; // Resolução da tela
/* Passo 0: */
XInitThreads();
/* Passo 1: */
display = XOpenDisplay(NULL);
if(display == NULL){
#if defined(W_DEBUG_WINDOW)
  fprintf(stderr, "ERROR: Failed to connect to X11 server.\n");
#endif
  return false; // Não conseguiu se conectar
}
/* Passo 2: */
_Wget_screen_resolution(&screen_resolution_x, &screen_resolution_y);
/* Passo 3: */
window = XCreateSimpleWindow(display, // Conexão com o X11
                             DefaultRootWindow(display), // Janela-mãe
                             0, 0, // Posição da janela criada
                             screen_resolution_x, // Largura
                             screen_resolution_y, // Altura
                             0, 0, // Espessura e cor da borda
                             0); // Cor padrão da janela
/* Passo 4: */
window_size_y =  screen_resolution_y;
#endif
@
\fimcodigo

Este código assume que temos as seguintes variável declarada para
armazenar e memorizar a janela criada:

\iniciocodigo
@<Variáveis Locais@>+=
#if !defined(_WIN32) && !defined(__EMSCRIPTEN__)
static Window window;    // Estrutura da janela criada
#endif
@
\fimcodigo

O código que temos até agora cria a janela. Mas não a desenha na
tela. Não desenhar ela na tela automaticamente permite que nós
ajustemos atributos da janela antes que ela seja finalmente exibida.

A primeira coisa que precisaremos ajustar é que queremos que a janela
seja em tela-cheia por padrão. Faremos isso pedindo para o gerenciador
de janelas não interferir na criação da janela, colocando bordas ou
tentando limitar seu tamanho que iremos ajustar. Mas faremos isso só
se realmente formos iniciar no modo tela-cheia:

\iniciocodigo
@<X11: Criar Janela@>+=
#if !defined(_WIN32) && !defined(__EMSCRIPTEN__)
#if !defined(W_WINDOW_NO_FULLSCREEN)
{
  XSetWindowAttributes attributes;
  attributes.override_redirect = true;
  XChangeWindowAttributes(display, window, CWOverrideRedirect,
                          &attributes);
}
#endif
#endif
@
\fimcodigo

Mas e se estivermos fora do modo de tela cheia e o usuário definiu
macros que dizem que o tamanho padrão da janela deve ser diferente de
ocupar a tela inteira? Neste caso, precisaremos redimencionar a janela
antes de desenhá-la na tela pela primeira vez. As macros que
controlarão o tamanho da janela quando não estamos em tela cheia
são \monoespaco{W\_WINDOW\_RESOLUTION\_X}
e \monoespaco{W\_WINDOW\_RESOLUTION\_Y}. Se elas valerem zero ou
menos, isso significa que o tamanho deve ser igual o da resolução da
tela. Caso contrário, seu valor representará o tamanho em pixels da
janela. Mas isso só se aplica quando não estamos em tela cheia:

\iniciocodigo
@<X11: Criar Janela@>+=
#if !defined(_WIN32) && !defined(__EMSCRIPTEN__)
#if defined(W_WINDOW_NO_FULLSCREEN)
{
  int window_size_x;
#if W_WINDOW_SIZE_X > 0
  window_size_x = W_WINDOW_SIZE_X;
#else
  window_size_x = screen_resolution_x;
#endif
#if W_WINDOW_SIZE_Y > 0
  window_size_y = W_WINDOW_SIZE_Y;
#else
  window_size_y = screen_resolution_y;
#endif
  XResizeWindow(display, window, window_size_x, window_size_y);  
}
#endif
#endif
@
\fimcodigo

Vamos também fixar o tamanho da janela para o atual para impedir que
ela de alguma forma seja redimencionada:

\iniciocodigo
@<X11: Criar Janela@>+=
#if !defined(_WIN32) && !defined(__EMSCRIPTEN__)
{
  XSizeHints hints;
  hints.flags = PMinSize | PMaxSize;
#if defined(W_WINDOW_NO_FULLSCREEN) && W_WINDOW_SIZE_X > 0
  hints.min_width = hints.max_width = W_WINDOW_SIZE_X;
#else
  hints.min_width = hints.max_width = screen_resolution_x;
#endif
#if defined(W_WINDOW_NO_FULLSCREEN) && W_WINDOW_SIZE_Y > 0
  hints.min_height = hints.max_height = W_WINDOW_SIZE_Y;
#else
  hints.min_height = hints.max_height = screen_resolution_y;
#endif
  XSetWMNormalHints(display, window, &hints);
}
#endif
@
\fimcodigo

O recurso acima requer o seguinte cabeçalho:

\iniciocodigo
@<Headers@>+=
#if !defined(_WIN32) && !defined(__EMSCRIPTEN__)
#include <X11/Xutil.h>
#endif
@
\fimcodigo

Outra coisa relevante a ajustar é em que tipo de eventos nosso
programa quer prestar atenção quando estiver executando. Exemplo de
evento que não consideraremos interessante: o usuário move a janela
pela tela. Exemplo de evento interessante: o usuário pressiona um
botão enquanto nossa janela está ativa.

A lista de eventos que considerearemos importantes o bastante para que
nosso programa seja notificado é: janela é criada ou destruída,
usuário aperta ou solta botão de teclado, usuário aperta ou solta
botão do mouse e usuário move o mouse. Se não pedirmos para sermos
informados disso, nenhum evento será informado para nosso programa e
ele não saberá quando o usuário interage com ele por meio de mouse e
teclado.

\iniciocodigo
@<X11: Criar Janela@>+=
#if !defined(_WIN32) && !defined(__EMSCRIPTEN__)
XSelectInput(display, window, StructureNotifyMask | KeyPressMask |
                              KeyReleaseMask | ButtonPressMask |
                              ButtonReleaseMask | PointerMotionMask);
#endif
@
\fimcodigo

Outra coisa importante é definir o nome da janela que será
criada. Geralmetne essa informação é apresentada de alguma forma pelo
gerenciador de janelas. Podemos deixar que o usuário escolha isso
ajustando a macro \monoespaco{W\_WINDOW\_NAME}:

\iniciocodigo
@<X11: Criar Janela@>+=
#if !defined(_WIN32) && !defined(__EMSCRIPTEN__)
XStoreName(display, window, W_WINDOW_NAME);
#endif
@
\fimcodigo

Se essta macro não estiver definida, deixamos apenas uma string vazia:

@<Define Macros@>=
#if !defined(W_WINDOW_NAME)
#define W_WINDOW_NAME ""
#endif
@
\fimcodigo

Agora vamos configurar o OpenGL ES. Como isso é suficientemente
trabalhoso, colocamos os passos de como fazer isso na próxima sessão:

\iniciocodigo
@<X11: Criar Janela@>+=
#if !defined(_WIN32) && !defined(__EMSCRIPTEN__)
@<X11: Configurar OpenGL ES@>
#endif
@
\fimcodigo

Uma vez que tenhamos ajustado as configurações de nossa janela,
podemos enfim começar a desenhar ela. Para isso enviamos uma
requisição para o servidor X e ficamos esperando em um laço até que
recebamos o evento de que a janela foi criada (já que pedimos para
sermos avisados deste evento passando a
flag \monoespaco{StructureNotifyMask} para a
função \monoespaco{XSelectInput} anteriormente). O código para isso é:

\iniciocodigo
@<X11: Criar Janela@>+=
#if !defined(_WIN32) && !defined(__EMSCRIPTEN__)
XMapWindow(display, window);
{
  XEvent e;
  do{
    XNextEvent(display, &e);
  } while(e.type != MapNotify);
}
#endif
@
\fimcodigo

\subsubsecao{2.2.2. Criando uma Janela com Web Assembly}

Um dos ambientes mais diferentes nos quais nossa API pode executar
será navegadores de Internet após ter o código compilado para Web
Assembly. Neste caso, não há janelas verdadeiras, o espaço no qual
poderemos desenhar na tela e teremos controle será um ``canvas'' de
HTML. Isso faz com que não tenhamos que nos preocupar com a
possibilidade do usuário tentar redimencionar a janela, por
exemplo. Mas ainda deveremos permitir ajustes no tamanho do nosso
``canvas'' além de continuarmos precisando prestar atenção no tamanho
da tela.

Aqui iremos manipular a nossa área de desenho combinando duas coisas:
a biblioteca SDL, fornecida como interface para realizar ações
gráficas pelo ambiente Emscripten e também código Javascript que
poderemos executar para ajudar.

Primeiro vamos inserir o cabeçalho do Emscripten com os cabeçalhos SDL:

\iniciocodigo
@<Cabeçalho OpenGL@>+=
#if defined(__EMSCRIPTEN__)
#include <GLES2/gl2.h>
#include <SDL/SDL.h>
#include <emscripten.h>
#endif
@
\fimcodigo

Agora vamos ler a resolução da tela com a função definida na Subsubseção 2.1.2:

\iniciocodigo
@<Web Assembly: Criar Janela@>=
#if defined(__EMSCRIPTEN__)
int screen_resolution_x, screen_resolution_y;
_Wget_screen_resolution(&screen_resolution_x, &screen_resolution_y);
#endif
@
\fimcodigo

A próxima coisa que temos a fazer é inicializar o sub-sistema de vídeo
da biblioteca SDL. Fazer isso é simplesmente chamar a função de
inicialização:

\iniciocodigo
@<Web Assembly: Criar Janela@>+=
#if defined(__EMSCRIPTEN__)
SDL_Init(SDL_INIT_VIDEO);
#endif
@
\fimcodigo

Agora vamos efetivamente criar a janela, o que na prática ajusta o
tamanho do canvas HTML onde iremos desenhar e o inicializa. O canvas
HTML deverá ter o tamanho da tela, exceto caso o usuário tenha
definido a macro \monoespaco{W\_WINDOW\_NO\_FULLSCREEN} e tenha
definido valores positivos para \monoespaco{W\_WINDOW\_RESOLUTION\_X}
e \monoespaco{W\_WINDOW\_RESOLUTION\_Y}. Também iremos nos certificar
de que o canvas está realmente visível, pois ele pode ter sido oculto
(é o que fazemos com ele se executarmos a função de fechar janela).

\iniciocodigo
@<Web Assembly: Criar Janela@>+=
#if defined(__EMSCRIPTEN__)
{
  int window_size_x = screen_resolution_x;
  int fullscreen_flag = SDL_WINDOW_FULLSCREEN;
  window_size_y = screen_resolution_y;
#if defined(W_WINDOW_NO_FULLSCREEN)
  fullscreen_flag = 0;
#if defined(W_WINDOW_SIZE_X) && W_Window_Size_X > 0
  window_size_x = W_Window_Size_X;
#endif
#if defined(W_WINDOW_SIZE_Y) && W_Window_Size_Y > 0
  window_size_y = W_Window_Size_Y;
#endif
#endif
  // Ajusta versão do OpenGL
  SDL_GL_SetAttribute(SDL_GL_CONTEXT_MAJOR_VERSION,
                     W_WINDOW_OPENGL_MAJOR_VERSION);
  SDL_GL_SetAttribute(SDL_GL_CONTEXT_MINOR_VERSION,
                     W_WINDOW_OPENGL_MINOR_VERSION);
  // Garante que o canvas estará visível
  EM_ASM(
    var el = document.getElementById("canvas");
    el.style.display = "initial";
  );
  window = SDL_SetVideoMode(window_size_x, window_size_y, 0,
                            SDL_OPENGL | fullscreen_flag);
  if(fullscreen_flag){
    EM_ASM(// Ativando tela-cheia no Javascript
      var el = document.getElementById("canvas");
      if(el.requestFullscreen){
        el.requestFullscreen();
      }
    );
  }
  if(window == NULL)
    return false;
}
#endif
@
\fimcodigo

Nossa janela neste caso é considerada uma superfície SDL:

\iniciocodigo
@<Variáveis Locais@>+=
#if defined(__EMSCRIPTEN__)
static SDL_Surface *window;
#endif
@
\fimcodigo

\subsubsecao{2.2.3. Criando uma Janela no Windows}

Como nos códigos anteriores, começamos lendo a resolução da tela:

\iniciocodigo
@<Windows: Criar Janela@>+=
#if defined(_WIN32)
int screen_resolution_x, screen_resolution_y;
_Wget_screen_resolution(&screen_resolution_x, &screen_resolution_y);
#endif
@
\fimcodigo

A próxima coisa a fazer é definir uma classe para a janela que iremos
criar. Primeiro vamos precisar dar um nome arbitrário para ela no
formato de uma string qualquer. Iremos chamá-la de ``WeaverWindow'':

\iniciocodigo
@<Variáveis Locais@>+=
#if defined(_WIN32)
static const char *class_name = "WeaverWindow";
#endif
@
\fimcodigo

Agora precisamos para nossa classe uma função que irá tratar todos os
sinais e mensagens enviada para nossa janela. Mensagens são enviadas e
devem ser tratadas quando a janela é criada, destruída,
redimencionada, exposta na tela, etc. Na dúvida sempre podemos
repassar cada mensagem para a função
padrão \monoespaco{DefWindowProc}, mas algumas coisas nós mesmos
teremos que definir. O formato da função que tratará as mensagens
recebidas pela janela é:

\iniciocodigo
@<Funções da API@>+=
#if defined(_WIN32)
LRESULT CALLBACK WindowProc(HWND window, UINT msg, WPARAM param1, LPARAM param2){
  switch(msg){
    @<Windows: Trata Mensagens para Janela@>
    default:
      return DefWindowProc(window, msg, param1, param2);
  }
}
#endif
@
\fimcodigo

Mas em quais casos iremos tratar as mensagens ao invés de apenas
passá-las adiante para o \monoespaco{DefWindowProc}? Um dos casos é
quando a janela receber uma mensagem para ser fechada:

\iniciocodigo
@<Windows: Trata Mensagens para Janela@>=
case WM_DESTROY:
  PostQuitMessage(0);
  return 0;
  break;
@
\fimcodigo

Agora temos que criar uma classe para a janela que iremos criar. Ela
deve apenas ter um nome único que não conflite com nomes padrão usados
pelo sistema. Também temos que passar na criação da classe um
identificador do programa que executamos (que obtemos
com \monoespaco{GetModuleHandle} e a função que irá lidar com
mensagens e sinais recebidos pela janela (no
caso, o padrão \monoespaco{DefWindowProc}).

\iniciocodigo
@<Windows: Criar Janela@>+=
#if defined(_WIN32)
if(!already_created_a_class){
  ATOM ret;
  WNDCLASS window_class;
  memset(&window_class, 0, sizeof(WNDCLASS));
  window_class.lpfnWndProc = WindowProc;
  window_class.hInstance = GetModuleHandle(NULL);
  window_class.lpszClassName = class_name;
  window_class.hbrBackground = CreateSolidBrush(RGB(0, 0, 0)); // Janela preta
  ret = RegisterClass(&window_class);
  if(ret == 0){
#if defined(W_DEBUG_WINDOW)
    fprintf(stderr, "ERROR: Failed to register Window Class. SysError: %d\n",
            GetLastError());
#endif
    return false;
  }
  already_created_a_class = true;
}
#endif
@
\fimcodigo

Por conveniência usamos a função \monoespaco{memset} para inicializar
a estrutura da classe da janela, já que a maior parte de seus
elementos pode ser mantida como zero, que é o padrão. Por isso,
inserimos o cabeçalho abaixo:

\iniciocodigo
@<Cabeçalhos@>+=
#if defined(_WIN32)
#include <string.h>
#endif
@
\fimcodigo

O código acima presume que temos declarada a seguinte variável que
armazena se nossa classe já foi criada:

\iniciocodigo
@<Variáveis Locais@>+=
#if defined(_WIN32)
static bool already_created_a_class = false;
#endif
@
\fimcodigo

Após termos a classe da janela, criamos a janela com o código
abaixo:

\iniciocodigo
@<Windows: Criar Janela@>+=
#if defined(_WIN32)
{
  DWORD fullscreen_flag = WS_POPUP;
  RECT size;
  int window_size_x = screen_resolution_x;
  window_size_y = screen_resolution_y;
  SystemParametersInfoA(SPI_GETWORKAREA, 0, &size, 0);
#if defined(W_WINDOW_NO_FULLSCREEN)
  window_size_x = size.left - size.right;
  window_size_y = size.bottom - size.top;
  fullscreen_flag = WS_OVERLAPPED;
#if defined(W_WINDOW_SIZE_X) && W_WINDOW_SIZE_X > 0
  window_size_x = W_WINDOW_SIZE_X;
#endif
#if defined(W_WINDOW_SIZE_Y) && W_WINDOW_SIZE_Y > 0
  window_size_y = W_WINDOW_SIZE_Y;
#endif
#endif
  window = CreateWindowEx(0, class_name,
                          W_WINDOW_NAME,
                          fullscreen_flag | WS_VISIBLE,
                          size.left, size.top, window_size_x,
                          window_size_y,
                          NULL, NULL,
                          GetModuleHandle(NULL),
                          NULL);
  if(window == NULL){
#if defined(W_DEBUG_WINDOW)
    fprintf(stderr, "ERROR: Failed creating window. SysCode: %d\n",
            GetLastError());
#endif
    return false;
  }
}
#endif
@
\fimcodigo

Iremos armazenar o identificador da janela criada na seguinte
variável:

\iniciocodigo
@<Variáveis Locais@>+=
#if defined(_WIN32)
static HWND window;
#endif
@
\fimcodigo

Antes de exibir a janela na tela, vamos configurar o OpenGL para
funcionar nela:

\iniciocodigo
@<Windows: Criar Janela@>+=
#if defined(_WIN32)
@<Windows: Configurar OpenGL@>
#endif
@
\fimcodigo

Agora pedimos para que o sistema passe a exibir a janela e esperamos
em um laço até o sistema avisar por meio de uma mensagem que a janela
foi criada:

\iniciocodigo
@<Windows: Criar Janela@>+=
#if defined(_WIN32)
{
  MSG msg;
  ShowWindow(window, SW_NORMAL);
  do{
    GetMessage(&msg, NULL, 0, 0);
  } while(msg.message == WM_CREATE);
}
#endif
@
\fimcodigo

\subsubsecao{2.2.4. Definindo a Função de Criação de Janelas para Todo Ambiente}

Nas subsubseções anteriores, definimos código para a criação de janela
em diferentes tipos de ambiente. E cada um dos códigos apresentado
ficava dentro de macros condicionais para ser executado somente no
ambiente correto.Agora podemos unir todos estes trechos de código
anteriores em uma só função:

\iniciocodigo
@<Funções da API@>=
bool _Wcreate_window(void){
  if(already_have_window == true)
    return false;
  @<X11: Criar Janela@>
  @<Web Assembly: Criar Janela@>
  @<Windows: Criar Janela@>
  _Wflush_window_input();
  already_have_window = true;
  return true;
}
@
\fimcodigo

A variável que armazena se a janela já está criada será declarada
aqui:

\iniciocodigo
@<Variáveis Locais@>+=
static bool already_have_window = false;
@
\fimcodigo

\subsecao{2.3. Configurando o OpenGL}

Nossa API deve suportar pelo menos o OpenG ES 2.0. Na maioria dos
ambientes isso não é um problema. Mas no Windows, por exemplo, nós não
temos garantias de que o sistema suporta OpenGL ES. Em tais casos,
devemos ativar o OpenGL 4, mas iremos suportar não a sua API inteira,
mas apenas a funções que estão prsentes no OpenGL ES 2.0.

\subsubsecao{2.3.1. Configurando OpenGL ES no X11}

No X11, a interface por meio da qual programamos usando OpenGL ES se
chama EGL e suas funções e macros são declaradas no seguinte
cabeçalho:

\iniciocodigo
@<Cabeçalho OpenGL@>+=
#if !defined(_WIN32) && !defined(__EMSCRIPTEN__)
#include <EGL/egl.h>
#include <GLES2/gl2.h>
#include <EGL/eglext.h>
#endif
@
\fimcodigo

Agora precisamos criar uma estrutura que armazena as informações sobre
nossa interface gráfica para o OpenGL ES. Assim como com o servidor X,
isso é armazenado em uma estrutura chamada \monoespaco{display}. E
podemos obter ela à partir do \monoespaco{display} da biblioteca X:

\iniciocodigo
@<X11: Configurar OpenGL ES@>+=
egl_display = eglGetPlatformDisplay(EGL_PLATFORM_X11_KHR, display,
                                    NULL);
if(egl_display == EGL_NO_DISPLAY){
#if defined(W_DEBUG_WINDOW)
  fprintf(stderr, "ERROR: Could not create EGL display.\n");
#endif
  return false;
}
eglInitialize(egl_display, NULL, NULL);
@
\fimcodigo

Essa variável é declarada aqui:

\iniciocodigo
@<Variáveis Locais@>+=
#if !defined(_WIN32) && !defined(__EMSCRIPTEN__)
static EGLDisplay *egl_display;
#endif
@
\fimcodigo

Agora obtemos uma configuração possível para o contexto OpenGL ES a
ser criado. Primeiro especificamos uma série de exigências e em
seguida obtemos da biblioteca uma configuração possível:

\iniciocodigo
@<X11: Configurar OpenGL ES@>+=
{
  bool ret;
  int number_of_configs_returned;
  int requested_attributes[] = {
    // Devemos suportar desenhar em janelas e em texturas:
    EGL_SURFACE_TYPE,  EGL_WINDOW_BIT | EGL_PBUFFER_BIT,
    // Devemos suportar ao menos 1 bit para a cor vermelha:
    EGL_RED_SIZE, 1,
    // Devemos suportar ao menos 1 bit para a cor verde:
    EGL_GREEN_SIZE, 1,
    // Devemos suportar ao menos 1 bit para a cor azul:
    EGL_BLUE_SIZE, 1,
    // Devemos suportar ao menos 1 bit para o canal alfa:
    EGL_ALPHA_SIZE, 1,
    // Devemos suportar ao menos 1 bit para a profundidade:
    EGL_DEPTH_SIZE, 1,
    EGL_NONE
  };
  ret = eglChooseConfig(egl_display, requested_attributes,
                        &egl_config, 1, &number_of_configs_returned);
  if(ret == EGL_FALSE){
#if defined(W_DEBUG_WINDOW)
    fprintf(stderr, "ERROR: Could not create valid EGL config.\n");
#endif
    return false;
  }
}
@
\fimcodigo

A estrutura que armazena a configuração que usaremos é declarada aqui:

\iniciocodigo
@<Variáveis Locais@>+=
#if !defined(_WIN32) && !defined(__EMSCRIPTEN__)
EGLConfig egl_config;
#endif
@
\fimcodigo


Assim como o EGL precisa de sua própria estrutura de display, ele
também precisará de uma estrutura própria para armazenar informações
sobre a janela na qual iremos desenhar. Podemos inicializar a
estrutura EGL para a janela à partir da estrutura de janela do X:

\iniciocodigo
@<X11: Configurar OpenGL ES@>+=
egl_window = eglCreateWindowSurface(egl_display, egl_config, window,
                                    NULL);
if(egl_window == EGL_NO_SURFACE){
#if defined(W_DEBUG_WINDOW)
  fprintf(stderr, "ERROR: Could not create EGL window.\n");
#endif
  return false;
}
@
\fimcodigo

A janela EGL é declarada aqui:

\iniciocodigo
@<Variáveis Locais@>+=
#if !defined(_WIN32) && !defined(__EMSCRIPTEN__)
static EGLSurface egl_window;
#endif
@
\fimcodigo


Agora criaremos o contexto OpenGL ES. Deixaremos que o usuário escolha
qual a versão do OpenGL por meio das
macros \monoespaco{W\_WINDOW\_OPENGL\_MAJOR\_VERSION} e
\monoespaco{W\_WINDOW\_OPENGL\_MINOR\_VERSION}. Usando tal informação, o código abaixo cria o contexto:

\iniciocodigo
@<X11: Configurar OpenGL ES@>+=
{
  int context_attribs[] = {
    EGL_CONTEXT_MAJOR_VERSION, W_WINDOW_OPENGL_MAJOR_VERSION,
    EGL_CONTEXT_MINOR_VERSION, W_WINDOW_OPENGL_MINOR_VERSION,
    EGL_NONE
  };
  egl_context = eglCreateContext(egl_display, egl_config,
                                 EGL_NO_CONTEXT, context_attribs);
  if(egl_context == EGL_NO_CONTEXT){
#if defined(W_DEBUG_WINDOW)
    fprintf(stderr, "ERROR: Could not create EGL context.\n");
#endif
    return false;
  }
  eglMakeCurrent(egl_display, egl_window, egl_window, egl_context);
}
@
\fimcodigo

O contexto OpenGL é declarado aqui:

\iniciocodigo
@<Variáveis Locais@>+=
#if !defined(_WIN32) && !defined(__EMSCRIPTEN__)
static EGLContext egl_context;
#endif
@
\fimcodigo

\subsubsecao{2.3.2.Configuring OpenGL on Web Assembly}

Nenhuma configuração adicional é necessária. Nós já suportamos OpenGL
porque passamos uma flag que pediu o suporte quando criamos uma
``janela'' na Subsubseção 2.2.2.

Note que a versção do OpenGL neste caso é o WebGL, mas esta versão é
compatível com o OpenGL ES.

\subsubsecao{2.3.3. Configurando OpenGL no Windows}

Para usar o OpenGL no Windows, primeiro precisamos avisamos o
compilador das bibliotecas necessárias para não precisar declará-las
durante o processo de ligação do programa e em seguida inserimos o
cabeçalho padrão e o do OpenGL:

\iniciocodigo
@<Cabeçalho OpenGL@>+=
#if defined(_WIN32)
#pragma comment(lib, "Opengl32.lib")
#pragma comment(lib, "User32.lib")
#pragma comment(lib, "Gdi32.lib")
#include <windows.h>
#include <GL/gl.h>
#endif
@
\fimcodigo

Também precisaremos de uma estrutura com informações sobre o
dispositivo em que iremos desenhar. No nosso caso, uma janela:

\iniciocodigo
@<Windows: Configurar OpenGL@>=
device_context = GetDC(window);
@
\fimcodigo

Esta estrutura é declarada aqui junto com o contexto OpenGL a ser
inicializado:

\iniciocodigo
@<Variáveis Locais@>+=
#if defined(_WIN32)
static HGLRC wgl_context;
static HDC device_context;
#endif
@
\fimcodigo


Além disso, configurar o OpenGL no Windows é uma tarefa um bocado
trabahosa. Pra começar, a função que cria contexto OpenGL definida por
padrão, a \monoespaco{wglCreateContext} pode criar um contexto muito
primitivo, sem suporte à funções mais recentes OpenGL e não há opção
para configurá-la com muitos dos parâmetros necessários para usar
recursos mais novos. Mas existe definida na prática uma outra função
que cria contexto: \monoespaco{wglCreateContextAttribsARB}, a qual
permite que façamos coisas básicas como pedir por uma versão
específica do OpenGL com suporte à funções mais modernas e a mais
parâmetros.

O problema é que a função \monoespaco{wglCreateContextAttribsARB} não
faz parte da API padrão e é considerada uma extensão. Então, para
podermos criar um contexto OpenGL adequado, precisamos carregar esta
função primeiro. Por outro lado, para usar a função que carrega
extensões, um contexto OpenGL já deve estar criado.

A forma de resolver isso é primeiro criar um contexto OpenGL básico e
primitivo suportado pela API, depois carregar a função de criação de
contexto moderno, criar o novo contexto, associar o contexto moderno à
uma janela e carregar como extensões as funções que precisamos. Mas o
problema é que não é possível carregar mais de um contexto OpenGL por
janela. Então, para fazer isso, precisamos usar uma janela descartável
e temporária na qual criaremos o contexto temporário e descartável:

\iniciocodigo
@<Windows: Configurar OpenGL@>+=
{
@<Windows: Criar uma Janela Temporária@>
@<Windows: Criar um Contexto Temporário@>
@<Windows: Carregar Funções OpenGL Iniciais@>
@<Windows: Remover Contexto e Janela Temporários@>
}
@
\fimcodigo

Primeiro vamos criar nossa janela descartável. Como só usaremos ela
para inicializar o OpenGL, não há necesidade de criar ela seguindo
todas as especificações de nossa janela verdadeira:

\iniciocodigo
@<Windows: Criar uma Janela Temporária@>=
HWND dummy_window;
{
  WNDCLASS dummy_window_class;
  memset(&dummy_window_class, 0, sizeof(WNDCLASS));
  dummy_window_class.lpfnWndProc = WindowProc;
  dummy_window_class.hInstance = GetModuleHandle(NULL);
  dummy_window_class.lpszClassName = "DummyWindow";
  // Esta função pode falhar se a classe já está registrada. Isso ocorre
  // quando a função que cria janelas é invocada mais de uma vez. Apenas
  // ignoramos os erros ao invocar a função abaixo:
  RegisterClass(&dummy_window_class);
  SetLastError(0);
  dummy_window = CreateWindowEx(0, dummy_window_class.lpszClassName, "Dummy",
                                0, CW_USEDEFAULT, CW_USEDEFAULT, CW_USEDEFAULT,
                                CW_USEDEFAULT, 0, 0,
                                dummy_window_class.hInstance, 0);
  if(dummy_window == NULL){
#if defined(W_DEBUG_WINDOW)
    fprintf(stderr, "ERROR: Failed creating window. SysCode: %d\n",
            GetLastError());
#endif
    return false;
  }
}
@
\fimcodigo

Agora vamos criar o contexto OpenGL temporário. Primeiro começamos
obtendo o contexto de dispositivo e configurando o formato de pixel
dele:

\iniciocodigo
@<Windows: Criar um Contexto Temporário@>=
HGLRC dummy_context;
HDC dummy_device_context = GetDC(dummy_window);
{
  PIXELFORMATDESCRIPTOR pixel_format;
  int chosen_pixel_format;
  memset(&pixel_format, 0, sizeof(WNDCLASS));
  pixel_format.nSize = sizeof(PIXELFORMATDESCRIPTOR); // Tamanho da estrutura
  pixel_format.nVersion = 1; // Número de versão
  pixel_format.dwFlags = PFD_DRAW_TO_WINDOW | PFD_SUPPORT_OPENGL |
                         PFD_DOUBLEBUFFER | PFD_DRAW_TO_BITMAP;
  pixel_format.iPixelType = PFD_TYPE_RGBA;
  pixel_format.cColorBits = 24; // 24 bits para profundidade de cor
  pixel_format.cDepthBits = 32; // 32 bits para buffer de profundidade
  pixel_format.iLayerType = PFD_MAIN_PLANE;
  chosen_pixel_format = ChoosePixelFormat(dummy_device_context, &pixel_format);
  if(chosen_pixel_format == 0){
#if defined(W_DEBUG_WINDOW)
    fprintf(stderr, "ERROR: Failed to choose a pixel format. SysError: %d\n",
            GetLastError());
#endif
    return false;
  }
  if(! SetPixelFormat(dummy_device_context, chosen_pixel_format, &pixel_format)){
#if defined(W_DEBUG_WINDOW)
    fprintf(stderr, "ERROR: Failed to set a pixel format. SysError: %d\n",
            GetLastError());
#endif
    return false;
  }
  // ...
@
\fimcodigo

Após termos configurado o formato de pixel que queremos, podemos obter
o contexto OpenGL temporário que queríamos:

\iniciocodigo
@<Windows: Criar um Contexto Temporário@>+=
  // ...
  dummy_context = wglCreateContext(dummy_device_context);
  if(dummy_context == NULL){
#if defined(W_DEBUG_WINDOW)
    fprintf(stderr, "ERROR: Failed creating dummy OpenGL context. SysError: %d\n",
            GetLastError());
#endif
    return false;
  }
  if(! wglMakeCurrent(dummy_device_context, dummy_context)){
#if defined(W_DEBUG_WINDOW)
    fprintf(stderr, "ERROR: Failed setting dummy OpenGL context. SysError: %d\n",
            GetLastError());
#endif
    return false;
  }
}
@
\fimcodigo

Agora temos que carregar as funções que queremos. Carregar uma função
existente, mas que não está declarada e acessível por ser considerada
uma extensão, é feito pela função definida abaixo que usa
\monoespaco{wglGetProcAddress} para obter um ponteiro para a função desejada:

\iniciocodigo
@<Funções Locais@>+=
#if defined(_WIN32)
static void *load_function(const char *name){
  void *ret = wglGetProcAddress(name);
  if(ret == NULL || ret == (void *) -1 || ret == (void *) 0x1 ||
     ret == (void *) 0x2 || ret == (void *) 0x3){
#if defined(W_DEBUG_WINDOW)
    fprintf(stderr, "ERROR: Function '%s' not supported.\n", name);
#endif
    return NULL;
  }
  return ret;
}
#endif
@
\fimcodigo

Note que como indicado pelo código acima, a
função \monoespaco{wglGetProcAddress} na prática pode indicar erro ou
falha de carregamento retornando 5 valores diferentes. Embora somente
o retorno de \monoespaco{NULL} seja realmente documentado como
correto.

As duas funções que precisamos carregar aqui é uma função com mais
recusos para escolher um formato de pixel (como o que escolhemos
antes, mas com suporte a mais parâmetros) e uma funcção para criar um
contexto OpenGL (como o que foi criado, mas também com mais recursos).

Para carregar novas funções, primeiro precisamos declarar ponteiros
com a posição de memória onde a nova função carregada será
armazenada. No caso das duas novas funções que queremos, declaramos o
ponteiro delas no cabeçalho \monoespaco{window.h} com o código abaixo:

\iniciocodigo
@<Declarações de Janela@>+=
#if defined(_WIN32)
extern BOOL (__stdcall *wglChoosePixelFormatARB)(HDC, const int *, const FLOAT *,
                                                 UINT, int *, UINT *);
extern HGLRC (*wglCreateContextAttribsARB)(HDC, HGLRC, const int *);
#endif
@
\fimcodigo

Também colocamos a declaração real no arquivo \monoespaco{window.c}:

\iniciocodigo
@<Variáveis Globais@>=
#if defined(_WIN32)
BOOL (__stdcall *wglChoosePixelFormatARB)(HDC, const int *, const FLOAT *, UINT,
                                          int *, UINT *);
HGLRC (*wglCreateContextAttribsARB)(HDC, HGLRC, const int *);
#endif
@
\fimcodigo

Uma vez que tenhamos a declaração dos ponteiros, podemos
inicializá-los carregando para eles as funções nas quais estamos
interessados:

\iniciocodigo
@<Windows: Carregar Funções OpenGL Iniciais@>+=
wglChoosePixelFormatARB = (BOOL (__stdcall *)(HDC, const int *, const FLOAT *,
                                              UINT, int *, UINT *))
                          load_function("wglChoosePixelFormatARB");
if(wglChoosePixelFormatARB == NULL) return false;
wglCreateContextAttribsARB = (HGLRC (*)(HDC, HGLRC, const int *))
                               load_function("wglCreateContextAttribsARB");
if(wglCreateContextAttribsARB == NULL) return false;
@
\fimcodigo

E finalmente, após termos carregado as duas funções acima, não temos
mais nenhuma necessidade da janela e do contexto temporários:

\iniciocodigo
@<Windows: Remover Contexto e Janela Temporários@>=
wglMakeCurrent(dummy_device_context, 0);
wglDeleteContext(dummy_context);
ReleaseDC(dummy_window, dummy_device_context);
DestroyWindow(dummy_window);
@
\fimcodigo

Agora estamos prontos para escolher o formato de pixel (a configuração
do OpenGL) da forma moderna com nossa nova função:

\iniciocodigo
@<Windows: Configurar OpenGL@>+=
{
  PIXELFORMATDESCRIPTOR pixel_format_descriptor;
  const int pixel_format_attributes[] = {
    WGL_DRAW_TO_WINDOW_ARB, GL_TRUE,
    WGL_SUPPORT_OPENGL_ARB, GL_TRUE,
    WGL_DOUBLE_BUFFER_ARB, GL_TRUE,
    WGL_ACCELERATION_ARB, WGL_FULL_ACCELERATION_ARB,
    WGL_PIXEL_TYPE_ARB, WGL_TYPE_RGBA_ARB,
    WGL_COLOR_BITS_ARB, 32,
    WGL_DEPTH_BITS_ARB, 24,
    WGL_STENCIL_BITS_ARB, 8,
    0 };
  int pixel_format_index = 0;
  UINT number_of_formats = 0;
  if(!wglChoosePixelFormatARB(device_context, pixel_format_attributes, NULL, 1,
                              &pixel_format_index,
                              (UINT *) (&number_of_formats))){
#if defined(W_DEBUG_WINDOW)
     fprintf(stderr, "ERROR: 'wglChoosePixelFormatARB' failed.\n");
#endif
     return false;
  }
  if(number_of_formats == 0){
#if defined(W_DEBUG_WINDOW)
     fprintf(stderr,
             "ERROR: no pixel format returned by 'wglChoosePixelFormatARB'.\n");
#endif
     return false;
  }
  DescribePixelFormat(device_context, pixel_format_index,
                      sizeof(pixel_format_descriptor), &pixel_format_descriptor);
  if(!SetPixelFormat(device_context, pixel_format_index,
                     &pixel_format_descriptor)){
#if defined(W_DEBUG_WINDOW)
    fprintf(stderr, "ERROR: 'SetPixelFormat' failed.\n");
#endif
    return false;
  }
}
@
\fimcodigo

Agora vamos especificar que queremos criar um contexto OpenGL cuja
versão do OpenGL é definida pelas macros que usamos para escolher a
versão. E em seguida usamos nossa função de criação de contexto
moderno:

\iniciocodigo
@<Windows: Configurar OpenGL@>+=
{
  const int opengl_attributes[] = {
    WGL_CONTEXT_MAJOR_VERSION_ARB, W_WINDOW_OPENGL_MAJOR_VERSION,
    WGL_CONTEXT_MINOR_VERSION_ARB, W_WINDOW_OPENGL_MINOR_VERSION,
    WGL_CONTEXT_FLAGS_ARB, WGL_CONTEXT_FORWARD_COMPATIBLE_BIT_ARB,
    0 };
  wgl_context = wglCreateContextAttribsARB(device_context, 0, opengl_attributes);
  if(wgl_context == NULL){
#if defined(W_DEBUG_WINDOW)
    fprintf(stderr, "ERROR: 'wglCreateContextAttribsARB' failed.\n");
#endif
    return false;
  }
  if(!wglMakeCurrent(device_context, wgl_context)){
#if defined(W_DEBUG_WINDOW)
    fprintf(stderr, "ERROR: 'wglMakeCurrent' failed.\n");
#endif
    return false;
  }
}
@
\fimcodigo

Durante a criação de nosso contexto OpenGL verdadeiro, fizemos uso de
uma série de macros que por padrão não estão definidas. Como o uso
delas é local, declaramos elas no cabeçalho do \monoespaco{window.c}:

\iniciocodigo
@<Cabeçalhos@>+=
#define WGL_TYPE_RGBA_ARB                      0x202B
#define WGL_PIXEL_TYPE_ARB                     0x2013
#define WGL_COLOR_BITS_ARB                     0x2014
#define WGL_DEPTH_BITS_ARB                     0x2022
#define WGL_STENCIL_BITS_ARB                   0x2023
#define WGL_ACCELERATION_ARB                   0x2003
#define WGL_DOUBLE_BUFFER_ARB                  0x2011
#define WGL_CONTEXT_FLAGS_ARB                  0x2094
#define WGL_DRAW_TO_WINDOW_ARB                 0x2001
#define WGL_SUPPORT_OPENGL_ARB                 0x2010
#define WGL_FULL_ACCELERATION_ARB              0x2027
#define WGL_CONTEXT_MAJOR_VERSION_ARB          0x2091
#define WGL_CONTEXT_MINOR_VERSION_ARB          0x2092
#define WGL_CONTEXT_FORWARD_COMPATIBLE_BIT_ARB 0x0002
@
\fimcodigo

Mas queremos suportar com nossa API as funções presentes no OpenGL ES
2.0. E tais funções também não fazem parte da API padrão oferecida
pelo WGL no Windows. O que faremos então é carregar tais funções como
extensões. Por exemplo, começando com as funções relacionadas à
criação e gerenciamento de shaders, vamos declarar seus ponteiros:

\iniciocodigo
@<Declarações de Janela@>+=
#if defined(_WIN32)
extern GLuint (__stdcall *glCreateShader)(GLenum shaderType);
extern void (__stdcall *glShaderSource)(GLuint, GLsizei, const GLchar *const*,
                                        const GLint *);
extern void (__stdcall *glCompileShader)(GLuint);
extern void (__stdcall *glReleaseShaderCompiler)(void);
extern void (__stdcall *glDeleteShader)(GLuint);
#endif
@
\fimcodigo

Em seguida posicionamos os ponteiros como variáveis globais em nosso
arquivo \monoespaco{window.c}:

\iniciocodigo
@<Variáveis Globais@>=
#if defined(_WIN32)
GLuint (__stdcall *glCreateShader)(GLenum shaderType);
void (__stdcall *glShaderSource)(GLuint, GLsizei, const GLchar *const*,
                                 const GLint *);
void (__stdcall *glCompileShader)(GLuint);
void (__stdcall *glReleaseShaderCompiler)(void);
void (__stdcall *glDeleteShader)(GLuint);
#endif
@
\fimcodigo

Em seguida carregamos para cada um dos ponteiros a função
correspondente usando a função definida um pouco acima de nossa
declaração de ponteiros:

\iniciocodigo
@<Windows: Configurar OpenGL@>+=
glCreateShader = (GLuint (__stdcall *)(GLenum)) load_function("glCreateShader");
if(glCreateShader == NULL)
  return false;
glShaderSource = (void (__stdcall *)(GLuint, GLsizei, const GLchar *const*,
                                     const GLint *))
                 load_function("glShaderSource");
if(glShaderSource == NULL)
  return false;
glCompileShader = (void (__stdcall *)(GLuint)) load_function("glCompileShader");
if(glCompileShader == NULL)
  return false;
glReleaseShaderCompiler = (void (__stdcall *)(void))
                             load_function("glReleaseShaderCompiler");
if(glReleaseShaderCompiler == NULL)
  return false;
glDeleteShader = (void (__stdcall *)(GLuint)) load_function("glDeleteShader");
if(glDeleteShader == NULL)
  return false;
@
\fimcodigo

Quando usamos \monoespaco{glCreateShader}, precisamos passar uma
destas macros como argumento para escolher o tipo de shader criado:

\iniciocodigo
@<Define Macros@>+=
#if defined(_WIN32)
#define GL_VERTEX_SHADER          0x8B31
#define GL_FRAGMENT_SHADER        0x8B30
#endif
@
\fimcodigo

O tipo \monoespaco{GLchar} também precisa ser criado no Windows:

\iniciocodigo
@<Define Macros@>+=
#if defined(_WIN32)
typedef char  GLchar;
#endif
@
\fimcodigo

Após compilar um shader, a ação típica desempenhada é checar se a
compilação fo bem-sucedida. Isso é feito usando funções que fazem
consultas com relação à shaders. Em particular, a
função \monoespaco{glGetShaderiv}. Declaramos abaixo o ponteiro das
funções relacionadas à consultas a shaders:

\iniciocodigo
@<Declarações de Janela@>+=
#if defined(_WIN32)
extern GLboolean (__stdcall *glIsShader)(GLuint);
extern void (__stdcall *glGetShaderiv)(GLuint, GLenum, GLint *);
extern void (__stdcall *glGetAttachedShaders)(GLuint, GLsizei, GLsizei *,
                                              GLuint *);
extern void (__stdcall *glGetShaderInfoLog)(GLuint, GLsizei, GLsizei *, GLchar *);
extern void (__stdcall *glGetShaderSource)(GLuint, GLsizei, GLsizei *, GLchar *);
extern void (__stdcall *glGetShaderPrecisionFormat)(GLenum, GLenum, GLint *,
                                                    GLint *);
extern void (__stdcall *glGetVertexAttribfv)(GLuint, GLenum, GLfloat *);
extern void (__stdcall *glGetVertexAttribiv)(GLuint, GLenum, GLint *);
extern void (__stdcall *glGetVertexAttribPointerv)(GLuint, GLenum, void **);
extern void (__stdcall *glGetUniformfv)(GLuint, GLint, GLfloat *);
extern void (__stdcall *glGetUniformiv)(GLuint, GLint, GLint *);
#endif
@
\fimcodigo

E posicionamos os ponteiros aqui::

\iniciocodigo
@<Variáveis Globais@>+=
#if defined(_WIN32)
GLboolean (__stdcall *glIsShader)(GLuint);
void (__stdcall *glGetShaderiv)(GLuint, GLenum, GLint *);
void (__stdcall *glGetAttachedShaders)(GLuint, GLsizei, GLsizei *, GLuint *);
void (__stdcall *glGetShaderInfoLog)(GLuint, GLsizei, GLsizei *, GLchar *);
void (__stdcall *glGetShaderSource)(GLuint, GLsizei, GLsizei *, GLchar *);
void (__stdcall *glGetShaderPrecisionFormat)(GLenum, GLenum, GLint *, GLint *);
void (__stdcall *glGetVertexAttribfv)(GLuint, GLenum, GLfloat *);
void (__stdcall *glGetVertexAttribiv)(GLuint, GLenum, GLint *);
void (__stdcall *glGetVertexAttribPointerv)(GLuint, GLenum, void **);
void (__stdcall *glGetUniformfv)(GLuint, GLint, GLfloat *);
void (__stdcall *glGetUniformiv)(GLuint, GLint, GLint *);
#endif
@
\fimcodigo

Carregamos cada uma das funções a seus respectivos ponteiros com o
código:

\iniciocodigo
@<Windows: Configurar OpenGL@>+=
glIsShader = (GLboolean (__stdcall *)(GLuint)) load_function("glIsShader");
if(glIsShader == NULL) return false;
glGetShaderiv = (void (__stdcall *)(GLuint, GLenum, GLint *))
                   load_function("glGetShaderiv");
if(glGetShaderiv == NULL) return false;
glGetAttachedShaders = (void (__stdcall *)(GLuint, GLsizei, GLsizei *, GLuint *))
                         load_function("glGetAttachedShaders");
if(glGetAttachedShaders == NULL) return false;
glGetShaderInfoLog = (void (__stdcall *)(GLuint, GLsizei, GLsizei *, GLchar *))
                         load_function("glGetShaderInfoLog");
if(glGetShaderInfoLog == NULL)  return false;
glGetShaderSource = (void (__stdcall *)(GLuint, GLsizei, GLsizei *, GLchar *))
                       load_function("glGetShaderSource");
if(glGetShaderSource == NULL) return false;
glGetShaderPrecisionFormat = (void (__stdcall *)(GLenum, GLenum, GLint *,
                                                 GLint *))
                                load_function("glGetShaderPrecisionFormat");
if(glGetShaderPrecisionFormat == NULL) return false;
glGetVertexAttribfv = (void (__stdcall *)(GLuint, GLenum, GLfloat *))
                         load_function("glGetVertexAttribfv");
if(glGetVertexAttribfv == NULL) return false;
glGetVertexAttribiv = (void (__stdcall *)(GLuint, GLenum, GLint *))
                        load_function("glGetVertexAttribiv");
if(glGetVertexAttribiv == NULL) return false;
glGetVertexAttribPointerv = (void (__stdcall *)(GLuint, GLenum, void **))
                               load_function("glGetVertexAttribPointerv");
if(glGetVertexAttribPointerv == NULL) return false;
glGetUniformfv = (void (__stdcall *)(GLuint, GLint, GLfloat *))
                     load_function("glGetUniformfv");
if(glGetUniformfv == NULL) return false;
glGetUniformiv = (void (__stdcall *)(GLuint, GLint, GLint *))
                     load_function("glGetUniformiv");
if(glGetUniformiv == NULL) return false;
@
\fimcodigo

Quando a função \monoespaco{glGetShaderiv} é usada, podemos selecionar
qual informação sobre o shader estamos consultando passando uma das
macros abaixo:

\iniciocodigo
@<Define Macros@>+=
#if defined(_WIN32)
#define GL_SHADER_TYPE          0x8B4F
#define GL_DELETE_STATUS        0x8B80
#define GL_COMPILE_STATUS       0x8B81
#define GL_INFO_LOG_LENGTH      0x8B84
#define GL_SHADER_SOURCE_LENGTH 0x8B88
#endif
@
\fimcodigo

Quando a função \monoespaco{glGetShaderPrecisionFormat} é usada para
consultar a precisão de algum tipo, o tipo a ser consultado é definido
passando uma das seguintes macros:

\iniciocodigo
@<Define Macros@>+=
#if defined(_WIN32)
#define GL_LOW_FLOAT    0x8DF0
#define GL_MEDIUM_FLOAT 0x8DF1
#define GL_HIGH_FLOAT   0x8DF2
#define GL_LOW_INT      0x8DF3
#define GL_MEDIUM_INT   0x8DF4
#define GL_HIGH_INT     0x8DF5
#endif
@
\fimcodigo

Quando a função \monoespaco{glGetVertexAttribfv}
ou \monoespaco{glGetVertexAttribiv} é usada para obter informações
sobre vértices, o tipo de informação desejada é informada passando
como argumento uma destas macros abaixo:

\iniciocodigo
@<Define Macros@>+=
#if defined(_WIN32)
#define GL_VERTEX_ATTRIB_ARRAY_BUFFER_BINDING 0x889F
#define GL_VERTEX_ATTRIB_ARRAY_ENABLED        0x8622
#define GL_VERTEX_ATTRIB_ARRAY_SIZE           0x8623
#define GL_VERTEX_ATTRIB_ARRAY_STRIDE         0x8624
#define GL_VERTEX_ATTRIB_ARRAY_TYPE           0x8625
#define GL_VERTEX_ATTRIB_ARRAY_NORMALIZED     0x886A
#define GL_CURRENT_VERTEX_ATTRIB              0x8626
#endif
@
\fimcodigo

Já quando usamos a função \monoespaco{glGetVertexzAttribPointerv},
precisamos passar como um de seus argumentos a macro abaixo:

\iniciocodigo
@<Define Macros@>+=
#if defined(_WIN32)
#define GL_VERTEX_ATTRIB_ARRAY_POINTER 0x8645
#endif
@
\fimcodigo

Vamos definir também esta macro que serve para consultar qual
implementação do GLSL temos:

\iniciocodigo
@<Define Macros@>+=
#if defined(_WIN32)
#define GL_SHADING_LANGUAGE_VERSION 0x8B8C
#endif
@
\fimcodigo

Uma vez que criamos e compilamos shaders, geralmente é desejado criar
um programa, ligar os shaders a ele e passar a usá-lo. Para permitir
isso, vamos declarar o ponteiro para cada uma das funções responsáveis
por lidar com programas:

\iniciocodigo
@<Declarações de Janela@>+=
#if defined(_WIN32)
extern GLuint (__stdcall *glCreateProgram)(void);
extern void (__stdcall *glAttachShader)(GLuint, GLuint);
extern void (__stdcall *glDetachShader)(GLuint, GLuint);
extern void (__stdcall *glLinkProgram)(GLuint);
extern void (__stdcall *glUseProgram)(GLuint);
extern void (__stdcall *glDeleteProgram)(GLuint);
#endif
@
\fimcodigo

Após declarar, os ponteiros são efetivamente posicionados aqui:

\iniciocodigo
@<Variáveis Globais@>+=
#if defined(_WIN32)
GLuint (__stdcall *glCreateProgram)(void);
void (__stdcall *glAttachShader)(GLuint, GLuint);
void (__stdcall *glDetachShader)(GLuint, GLuint);
void (__stdcall *glLinkProgram)(GLuint);
void (__stdcall *glUseProgram)(GLuint);
void (__stdcall *glDeleteProgram)(GLuint);
#endif
@
\fimcodigo

E inicializamos os ponteiros com as funções adequadas:

\iniciocodigo
@<Windows: Configurar OpenGL@>+=
glCreateProgram = (GLuint (__stdcall *)(void)) load_function("glCreateProgram");
if(glCreateProgram == NULL) return false;
glAttachShader = (void (__stdcall *)(GLuint, GLuint))
                  load_function("glAttachShader");
if(glAttachShader == NULL) return false;
glDetachShader = (void (__stdcall *)(GLuint, GLuint))
                 load_function("glDetachShader");
if(glDetachShader == NULL) return false;
glLinkProgram = (void (__stdcall *)(GLuint)) load_function("glLinkProgram");
if(glLinkProgram == NULL) return false;
glUseProgram = (void (__stdcall *)(GLuint)) load_function("glUseProgram");
if(glUseProgram == NULL) return false;
glDeleteProgram = (void (__stdcall *)(GLuint)) load_function("glDeleteProgram");
if(glDeleteProgram == NULL) return false;
@
\fimcodigo

Terminada a geração de um programa GLSL, geralmente o usuário irá
checar se deu tudo certo na criação do programa. E para isso é
importante prepararmos as funções que fazem consultas a programas:

\iniciocodigo
@<Declarações de Janela@>+=
#if defined(_WIN32)
extern GLboolean (__stdcall *glIsProgram)(GLuint);
extern void (__stdcall *glGetProgramiv)(GLuint, GLenum, GLint *);
extern void (__stdcall *glGetProgramInfoLog)(GLuint, GLsizei, GLsizei *,
                                             GLchar *);
extern void (__stdcall *glValidadeProgram)(GLuint);
#endif
@
\fimcodigo

Posicionamos os ponteiros reais aqui:

\iniciocodigo
@<Variáveis Globais@>+=
#if defined(_WIN32)
GLboolean (__stdcall *glIsProgram)(GLuint);
void (__stdcall *glGetProgramiv)(GLuint, GLenum, GLint *);
void (__stdcall *glGetProgramInfoLog)(GLuint, GLsizei, GLsizei *, GLchar *);
void (__stdcall *glValidadeProgram)(GLuint);
#endif
@
\fimcodigo

E os inicializamos com as funções reais:

\iniciocodigo
@<Windows: Configurar OpenGL@>+=
glIsProgram = (GLboolean (__stdcall *)(GLuint)) load_function("glIsProgram");
if(glIsProgram == NULL) return false;
glGetProgramiv = (void (__stdcall *)(GLuint, GLenum, GLint *))
                    load_function("glGetProgramiv");
if(glGetProgramiv == NULL) return false;
glGetProgramInfoLog = (void (__stdcall *)(GLuint, GLsizei, GLsizei *, GLchar *))
                          load_function("glGetProgramInfoLog");
if(glGetProgramInfoLog == NULL) return false;
glValidadeProgram = (void (__stdcall *)(GLuint))
                     load_function("glValidateProgram");
if(glValidadeProgram == NULL) return false;
@
\fimcodigo

Quando usamos \monoespaco{glGetProgramiv} para obter informação sobre
um programa, passamos como argumento uma destas macros a seguir para
selecionar qual informação queremos saber:

\iniciocodigo
@<Define Macros@>+=
#if defined(_WIN32)
#define GL_DELETE_STATUS               0x8B80
#define GL_LINK_STATUS                 0x8B82
#define GL_VALIDATE_STATUS             0x8B83
#define GL_INFO_LOG_LENGTH             0x8B84
#define GL_ATTACHED_SHADERS            0x8B85
#define GL_ACTIVE_ATTRIBUTES           0x8B89
#define GL_ACTIVE_ATTRIBUTE_MAX_LENGTH 0x8B8A
#define GL_ACTIVE_UNIFORMS             0x8B86
#define GL_ACTIVE_UNIFORM_MAX_LENGTH   0x8B87
#endif
@
\fimcodigo

Já para obter e escolher aributos de vértices dentro de um shader,
usaremos funções que serão ligadas aos seguintes ponteiros:

\iniciocodigo
@<Declarações de Janela@>+=
#if defined(_WIN32)
extern void (__stdcall *glGetActiveAttrib)(GLuint, GLuint, GLsizei, GLsizei *,
                                           GLint *, GLenum *, GLchar *);
extern GLint (__stdcall *glGetAttribLocation)(GLuint, const GLchar *);
extern void (__stdcall *glBindAttribLocation)(GLuint, GLuint, const GLchar *);
#endif
@
\fimcodigo

Que serão posicionados aqui:

\iniciocodigo
@<Variáveis Globais@>+=
#if defined(_WIN32)
void (__stdcall *glGetActiveAttrib)(GLuint, GLuint, GLsizei, GLsizei *, GLint *,
                                    GLenum *, GLchar *);
GLint (__stdcall *glGetAttribLocation)(GLuint, const GLchar *);
void (__stdcall *glBindAttribLocation)(GLuint, GLuint, const GLchar *);
#endif
@
\fimcodigo

E eles são inicializados aqui:

\iniciocodigo
@<Windows: Configurar OpenGL@>+=
glGetActiveAttrib = (void (__stdcall *)(GLuint, GLuint, GLsizei, GLsizei *,
                                        GLint *, GLenum *, GLchar *))
                    load_function("glGetActiveAttrib");
if(glGetActiveAttrib == NULL) return false;
glGetAttribLocation = (GLint (__stdcall *)(GLuint, const GLchar *))
                           load_function("glGetAttribLocation");
if(glGetAttribLocation == NULL) return false;
glBindAttribLocation = (void (__stdcall *)(GLuint, GLuint, const GLchar *))
                              load_function("glBindAttribLocation");
if(glBindAttribLocation == NULL) return false;
@
\fimcodigo

O tipo de um atributo de vértice, que é retornado
por \monoespaco{glGetActiveAttrib} pode ser:

\iniciocodigo
@<Define Macros@>+=
#if defined(_WIN32)
#define GL_FLOAT      0x1406
#define GL_FLOAT_VEC2 0x8B50
#define GL_FLOAT_VEC3 0x8B51
#define GL_FLOAT_VEC4 0x8B52
#define GL_FLOAT_MAT2 0x8B5A
#define GL_FLOAT_MAT3 0x8B5B
#define GL_FLOAT_MAT4 0x8B5C
#endif
@
\fimcodigo

E finalmente, as últimas funções relacionadas aos shaders são as
responsáveis por lidar com variáveis uniformes:

\iniciocodigo
@<Declarações de Janela@>+=
#if defined(_WIN32)
extern GLint (__stdcall *glGetUniformLocation)(GLuint, const GLchar *);
extern void (__stdcall *glGetActiveUniform)(GLuint, GLuint, GLsizei, GLsizei *,
                                            GLint *, GLenum *, GLchar *);
extern void (__stdcall *glUniform1f)(GLint, GLfloat);
extern void (__stdcall *glUniform2f)(GLint, GLfloat, GLfloat);
extern void (__stdcall *glUniform3f)(GLint, GLfloat, GLfloat, GLfloat);
extern void (__stdcall *glUniform4f)(GLint, GLfloat, GLfloat, GLfloat, GLfloat);
extern void (__stdcall *glUniform1i)(GLint, GLint);
extern void (__stdcall *glUniform2i)(GLint, GLint, GLint);
extern void (__stdcall *glUniform3i)(GLint, GLint, GLint, GLint);
extern void (__stdcall *glUniform4i)(GLint, GLint, GLint, GLint, GLint);
extern void (__stdcall *glUniform1fv)(GLint, GLsizei, const GLfloat *);
extern void (__stdcall *glUniform2fv)(GLint, GLsizei, const GLfloat *);
extern void (__stdcall *glUniform3fv)(GLint, GLsizei, const GLfloat *);
extern void (__stdcall *glUniform4fv)(GLint, GLsizei, const GLfloat *);
extern void (__stdcall *glUniform1iv)(GLint, GLsizei, const GLint *);
extern void (__stdcall *glUniform2iv)(GLint, GLsizei, const GLint *);
extern void (__stdcall *glUniform3iv)(GLint, GLsizei, const GLint *);
extern void (__stdcall *glUniform4iv)(GLint, GLsizei, const GLint *);
extern void (__stdcall *glUniformMatrix2fv)(GLint, GLsizei, GLboolean,
                                            const GLfloat *);
extern void (__stdcall *glUniformMatrix3fv)(GLint, GLsizei, GLboolean,
                                            const GLfloat *);
extern void (__stdcall *glUniformMatrix4fv)(GLint, GLsizei, GLboolean,
                                            const GLfloat *);
#endif
@
\fimcodigo

Estes 21 ponteiros para funções são posicionados aqui:

\iniciocodigo
@<Variáveis Globais@>+=
#if defined(_WIN32)
GLint (__stdcall *glGetUniformLocation)(GLuint, const GLchar *);
void (__stdcall *glGetActiveUniform)(GLuint, GLuint, GLsizei, GLsizei *, GLint *,
                                     GLenum *, GLchar *);
void (__stdcall *glUniform1f)(GLint, GLfloat);
void (__stdcall *glUniform2f)(GLint, GLfloat, GLfloat);
void (__stdcall *glUniform3f)(GLint, GLfloat, GLfloat, GLfloat);
void (__stdcall *glUniform4f)(GLint, GLfloat, GLfloat, GLfloat, GLfloat);
void (__stdcall *glUniform1i)(GLint, GLint);
void (__stdcall *glUniform2i)(GLint, GLint, GLint);
void (__stdcall *glUniform3i)(GLint, GLint, GLint, GLint);
void (__stdcall *glUniform4i)(GLint, GLint, GLint, GLint, GLint);
void (__stdcall *glUniform1fv)(GLint, GLsizei, const GLfloat *);
void (__stdcall *glUniform2fv)(GLint, GLsizei, const GLfloat *);
void (__stdcall *glUniform3fv)(GLint, GLsizei, const GLfloat *);
void (__stdcall *glUniform4fv)(GLint, GLsizei, const GLfloat *);
void (__stdcall *glUniform1iv)(GLint, GLsizei, const GLint *);
void (__stdcall *glUniform2iv)(GLint, GLsizei, const GLint *);
void (__stdcall *glUniform3iv)(GLint, GLsizei, const GLint *);
void (__stdcall *glUniform4iv)(GLint, GLsizei, const GLint *);
void (__stdcall *glUniformMatrix2fv)(GLint, GLsizei, GLboolean, const GLfloat *);
void (__stdcall *glUniformMatrix3fv)(GLint, GLsizei, GLboolean, const GLfloat *);
void (__stdcall *glUniformMatrix4fv)(GLint, GLsizei, GLboolean, const GLfloat *);
#endif
@
\fimcodigo

E agora temos que inicializar todos estes ponteiros com suas funções:

\iniciocodigo
@<Windows: Configurar OpenGL@>+=
glGetUniformLocation = (GLint (__stdcall *)(GLuint, const GLchar *))
                            load_function("glGetUniformLocation");
if(glGetUniformLocation == NULL) return false;
glGetActiveUniform = (void (__stdcall *)(GLuint, GLuint, GLsizei, GLsizei *,
                                         GLint *, GLenum *, GLchar *))
                     load_function("glGetActiveUniform");
if(glGetActiveUniform == NULL) return false;
glUniform1f = (void (__stdcall *)(GLint, GLfloat)) load_function("glUniform1f");
if(glUniform1f == NULL) return false;
glUniform2f = (void (__stdcall *)(GLint, GLfloat, GLfloat))
               load_function("glUniform2f");
if(glUniform2f == NULL) return false;
glUniform3f = (void (__stdcall *)(GLint, GLfloat, GLfloat, GLfloat))
                        load_function("glUniform3f");
if(glUniform3f == NULL) return false;
glUniform4f = (void (__stdcall *)(GLint, GLfloat, GLfloat, GLfloat, GLfloat))
                        load_function("glUniform4f");
if(glUniform4f == NULL) return false;
glUniform1i = (void (__stdcall *)(GLint, GLint)) load_function("glUniform1i");
if(glUniform1i == NULL) return false;
glUniform2i = (void (__stdcall *)(GLint, GLint, GLint)) load_function("glUniform2i");
if(glUniform2i == NULL) return false;
glUniform3i = (void (__stdcall *)(GLint, GLint, GLint, GLint))
               load_function("glUniform3i");
if(glUniform3i == NULL) return false;
glUniform4i = (void (__stdcall *)(GLint, GLint, GLint, GLint, GLint))
                 load_function("glUniform4i");
if(glUniform4i == NULL) return false;
glUniform1fv = (void (__stdcall *)(GLint, GLsizei, const GLfloat *))
                 load_function("glUniform1fv");
if(glUniform1fv == NULL) return false;
glUniform2fv = (void (__stdcall *)(GLint, GLsizei, const GLfloat *))
                 load_function("glUniform2fv");
if(glUniform2fv == NULL) return false;
glUniform3fv = (void (__stdcall *)(GLint, GLsizei, const GLfloat *))
                 load_function("glUniform3fv");
if(glUniform3fv == NULL) return false;
glUniform4fv = (void (__stdcall *)(GLint, GLsizei, const GLfloat *))
                 load_function("glUniform4fv");
if(glUniform4fv == NULL) return false;
glUniform1iv = (void (__stdcall *)(GLint, GLsizei, const GLint *))
                 load_function("glUniform1iv");
if(glUniform1iv == NULL) return false;
glUniform2iv = (void (__stdcall *)(GLint, GLsizei, const GLint *))
                 load_function("glUniform2iv");
if(glUniform2iv == NULL) return false;
glUniform3iv = (void (__stdcall *)(GLint, GLsizei, const GLint *))
                 load_function("glUniform3iv");
if(glUniform3iv == NULL) return false;
glUniform4iv = (void (__stdcall *)(GLint, GLsizei, const GLint *))
                 load_function("glUniform4iv");
if(glUniform4iv == NULL) return false;
glUniformMatrix2fv = (void (__stdcall *)(GLint, GLsizei, GLboolean,
                                         const GLfloat *))
                      load_function("glUniformMatrix2fv");
if(glUniformMatrix2fv == NULL) return false;
glUniformMatrix3fv = (void (__stdcall *)(GLint, GLsizei, GLboolean,
                                         const GLfloat *))
                      load_function("glUniformMatrix3fv");
if(glUniformMatrix3fv == NULL) return false;
glUniformMatrix4fv = (void (__stdcall *)(GLint, GLsizei, GLboolean,
                                         const GLfloat *))
                      load_function("glUniformMatrix4fv");
if(glUniformMatrix4fv == NULL) return false;
@
\fimcodigo

As variávei uniformes podem ter o mesmo tipo que atributos de vértice,
mas podem ter também alguns destes novos tipos:

\iniciocodigo
@<Define Macros@>+=
#if defined(_WIN32)
#define GL_INT         0x1404
#define GL_INT_VEC2    0x8B53
#define GL_INT_VEC3    0x8B54
#define GL_INT_VEC4    0x8B55
#define GL_BOOL        0x8B56
#define GL_BOOL_VEC2   0x8B57
#define GL_BOOL_VEC3   0x8B58
#define GL_BOOL_VEC4   0x8B59
#define GL_SAMPLER_2D  0x8B5E
#define GL_SAMPER_CUBE 0x8B60
#endif
@
\fimcodigo

A última coisa que precisamos para renderizar imagens simples é enviar
vértices para a placa de vídeo. Uma das formas de fazer isso é enviar
ponteiro para os vértices na placa de vídeo ao invés de enviá-los
diretamente para acesso mais rápido. Essa opção pode ser feita com as
funções abaixo:

\iniciocodigo
@<Declarações de Janela@>+=
#if defined(_WIN32)
extern void (__stdcall *glVertexAttrib1f)(GLuint, GLfloat);
extern void (__stdcall *glVertexAttrib2f)(GLuint, GLfloat, GLfloat);
extern void (__stdcall *glVertexAttrib3f)(GLuint, GLfloat, GLfloat, GLfloat);
extern void (__stdcall *glVertexAttrib4f)(GLuint, GLfloat, GLfloat, GLfloat,
                                          GLfloat);
extern void (__stdcall *glVertexAttrib1fv)(GLuint, GLfloat *);
extern void (__stdcall *glVertexAttrib2fv)(GLuint, GLfloat *);
extern void (__stdcall *glVertexAttrib3fv)(GLuint, GLfloat *);
extern void (__stdcall *glVertexAttrib4fv)(GLuint, GLfloat *);
extern void (__stdcall *glVertexAttribPointer)(GLuint, GLint, GLenum, GLboolean,
                                               GLsizei, const void *);
extern void (__stdcall *glEnableVertexAttribArray)(GLuint);
extern void (__stdcall *glDisableVertexAttribArray)(GLuint);
#endif
@
\fimcodigo

Posicionamos os ponteiros como variáveis globais:

\iniciocodigo
@<Variáveis Globais@>+=
#if defined(_WIN32)
void (__stdcall *glVertexAttrib1f)(GLuint, GLfloat);
void (__stdcall *glVertexAttrib2f)(GLuint, GLfloat, GLfloat);
void (__stdcall *glVertexAttrib3f)(GLuint, GLfloat, GLfloat, GLfloat);
void (__stdcall *glVertexAttrib4f)(GLuint, GLfloat, GLfloat, GLfloat, GLfloat);
void (__stdcall *glVertexAttrib1fv)(GLuint, GLfloat *);
void (__stdcall *glVertexAttrib2fv)(GLuint, GLfloat *);
void (__stdcall *glVertexAttrib3fv)(GLuint, GLfloat *);
void (__stdcall *glVertexAttrib4fv)(GLuint, GLfloat *);
void (__stdcall *glVertexAttribPointer)(GLuint, GLint, GLenum, GLboolean,
                                        GLsizei, const void *);
void (__stdcall *glEnableVertexAttribArray)(GLuint);
void (__stdcall *glDisableVertexAttribArray)(GLuint);
#endif
@
\fimcodigo

E as inicializamos:

\iniciocodigo
@<Windows: Configurar OpenGL@>+=
glVertexAttrib1f = (void (__stdcall *)(GLuint, GLfloat))
                     load_function("glVertexAttrib1f");
if(glVertexAttrib1f == NULL) return false;
glVertexAttrib2f = (void (__stdcall *)(GLuint, GLfloat, GLfloat))
                      load_function("glVertexAttrib2f");
if(glVertexAttrib2f == NULL) return false;
glVertexAttrib3f = (void (__stdcall *)(GLuint, GLfloat, GLfloat, GLfloat))
                      load_function("glVertexAttrib3f");
if(glVertexAttrib3f == NULL) return false;
glVertexAttrib4f = (void (__stdcall *)(GLuint, GLfloat, GLfloat, GLfloat,
                                       GLfloat))
                      load_function("glVertexAttrib4f");
if(glVertexAttrib4f == NULL) return false;
glVertexAttrib1fv = (void (__stdcall *)(GLuint, GLfloat *))
                       load_function("glVertexAttrib1fv");
if(glVertexAttrib1fv == NULL) return false;
glVertexAttrib2fv = (void (__stdcall *)(GLuint, GLfloat *))
                       load_function("glVertexAttrib2fv");
if(glVertexAttrib2fv == NULL) return false;
glVertexAttrib3fv = (void (__stdcall *)(GLuint, GLfloat *))
                       load_function("glVertexAttrib3fv");
if(glVertexAttrib3fv == NULL) return false;
glVertexAttrib4fv = (void (__stdcall *)(GLuint, GLfloat *))
                       load_function("glVertexAttrib4fv");
if(glVertexAttrib4fv == NULL) return false;
glVertexAttribPointer = (void (__stdcall *)(GLuint, GLint, GLenum, GLboolean,
                         GLsizei, const void *))
                              load_function("glVertexAttribPointer");
if(glVertexAttribPointer == NULL) return false;
glEnableVertexAttribArray = (void (__stdcall *)(GLuint))
                              load_function("glEnableVertexAttribArray");
if(glEnableVertexAttribArray == NULL) return false;
glDisableVertexAttribArray = (void (__stdcall *)(GLuint))
                               load_function("glDisableVertexAttribArray");
if(glDisableVertexAttribArray == NULL) return false;
@
\fimcodigo

Os atributos de vétices podem ter tipos específicos. Além de tipos que
já foram definidos (\monoespaco{GL\_FLOAT}), é necessário definir
também:

\iniciocodigo
@<Define Macros@>+=
#if defined(_WIN32)
#define GL_FIXED          0x140C
#endif
@
\fimcodigo

Agora definiremos os ponteiros para as funções relacionadas a objetos
do tipo buffer. Os ponteiros são:

\iniciocodigo
@<Declarações de Janela@>+=
#if defined(_WIN32)
extern void (__stdcall *glGenBuffers)(GLsizei, GLuint *);
extern void (__stdcall *glDeleteBuffers)(GLsizei, const GLuint *);
extern void (__stdcall *glBindBuffer)(GLenum, GLuint);
extern void (__stdcall *glBufferData)(GLenum, GLsizeiptr, const void *, GLenum);
extern void (__stdcall *glBufferSubData)(GLenum, GLintptr, GLsizeiptr,
                                         const void *);
extern void (__stdcall *glIsBuffer)(GLuint);
extern void (__stdcall *glGetBufferParameteriv)(GLenum, GLenum, GLint *);
#endif
@
\fimcodigo

E a declaração:

\iniciocodigo
@<Variáveis Globais@>+=
#if defined(_WIN32)
void (__stdcall *glGenBuffers)(GLsizei, GLuint *);
void (__stdcall *glDeleteBuffers)(GLsizei, const GLuint *);
void (__stdcall *glBindBuffer)(GLenum, GLuint);
void (__stdcall *glBufferData)(GLenum, GLsizeiptr, const void *, GLenum);
void (__stdcall *glBufferSubData)(GLenum, GLintptr, GLsizeiptr, const void *);
void (__stdcall *glIsBuffer)(GLuint);
void (__stdcall *glGetBufferParameteriv)(GLenum, GLenum, GLint *);
#endif
@
\fimcodigo

A inicialização:

\iniciocodigo
@<Windows: Configurar OpenGL@>+=
glGenBuffers = (void (__stdcall *)(GLsizei, GLuint *))
                 load_function("glGenBuffers");
if(glGenBuffers == NULL) return false;
glDeleteBuffers = (void (__stdcall *)(GLsizei, const GLuint *))
                    load_function("glDeleteBuffers");
if(glDeleteBuffers == NULL) return false;
glBindBuffer = (void (__stdcall *)(GLenum, GLuint))load_function("glBindBuffer");
if(glBindBuffer == NULL) return false;
glBufferData = (void (__stdcall *)(GLenum, GLsizeiptr, const void *, GLenum))
                 load_function("glBufferData");
if(glBufferData == NULL) return false;
glBufferSubData = (void (__stdcall *)(GLenum, GLintptr, GLsizeiptr, const void *))
                    load_function("glBufferSubData");
if(glBufferSubData == NULL) return false;
glIsBuffer = (void (__stdcall *)(GLuint)) load_function("glIsBuffer");
if(glIsBuffer == NULL) return false;
glGetBufferParameteriv = (void (__stdcall *)(GLenum, GLenum, GLint *))
                           load_function("glGetBufferParameteriv");
if(glGetBufferParameteriv == NULL) return false;
@
\fimcodigo

Quando escolhemos ativar um buffer com \monoespaco{glBindBuffer}, ele
pode ser dos seguintes tipos:

\iniciocodigo
@<Define Macros@>+=
#if defined(_WIN32)
#define GL_ARRAY_BUFFER         0x8892
#define GL_ELEMENT_ARRAY_BUFFER 0x8893
#endif
@
\fimcodigo

Quando passamos dados em um buffer com \monoespaco{glBufferData},
podemos escolher dentre os seguintes modos de uso para os dados:

\iniciocodigo
@<Define Macros@>+=
#if defined(_WIN32)
#define GL_STATIC_DRAW  0x88E4
#define GL_STREAM_DRAW  0x88E0
#define GL_DYNAMIC_DRAW 0x88E8
#endif
@
\fimcodigo

Quando escolhemos pedir informações sobre um buffer
com \monoespaco{glGetBufferParameteriv}, a informação pedida pode ser:

\iniciocodigo
@<Define Macros@>+=
#if defined(_WIN32)
#define GL_BUFFER_SIZE  0x8764
#define GL_BUFFER_USAGE 0x8765
#endif
@
\fimcodigo

E as novas funções requerem também que definamos os tipos de dados
abaixo que devem ser grandes o bastante para armazenar ponteiros,
mesmo não necessariamente sendo ponteiros. A diferença entre eles é
somente ter ou não ter sinal.

\iniciocodigo
@<Define Macros@>+=
#if defined(_WIN32)
// Isso inclui um signed size_t:
#include <BaseTsd.h>
typedef size_t GLsizeiptr;
typedef SSIZE_T GLintptr;
#endif
@
\fimcodigo

As duas funções da API referentes à janela de exibição (``viewport'')
e recortte (``clipping'') são:

\iniciocodigo
@<Declarações de Janela@>+=
#if defined(_WIN32)
extern void (__stdcall *glDepthRangef)(GLclampf, GLclampf);
#endif
@
\fimcodigo

Suas declarações:

\iniciocodigo
@<Variáveis Globais@>+=
#if defined(_WIN32)
void (__stdcall *glDepthRangef)(GLclampf, GLclampf);
#endif
@
\fimcodigo

E inicialização:

\iniciocodigo
@<Windows: Configurar OpenGL@>+=
glDepthRangef = (void (__stdcall *)(GLclampf, GLclampf))
                   load_function("glDepthRangef");
if(glDepthRangef == NULL) return false;
@
\fimcodigo

O tipo \monoespaco{GLclampf} representa um número em ponto flutuante
que deve estar no intervalo entre 0 e 1. Podemos representá-los como
números em ponto flutuante mesmo:

\iniciocodigo
@<Define Macros@>+=
#if defined(_WIN32)
typedef float GLclampf;
#endif
@
\fimcodigo

Agora vamos preparar as funções do OpenGL relacionadas à textura:

\iniciocodigo
@<Declarações de Janela@>+=
#if defined(_WIN32)
extern void (__stdcall *glActiveTexture)(GLenum);
extern void (__stdcall *glCompressedTexImage2D)(GLenum, int, GLenum, GLsizei,
                                                GLsizei, int, GLsizei, void *);
extern void (__stdcall *glCompressedTexSubImage2D)(GLenum, int, int, int, GLsizei,
                                                   GLsizei, GLenum, GLsizei,
                                                   void *);
extern void (__stdcall *glGenerateMipmap)(GLenum);
#endif
@
\fimcodigo

E as declaramos no arquivo \monoespaco{window.c}:

\iniciocodigo
@<Variáveis Globais@>+=
#if defined(_WIN32)
void (__stdcall *glActiveTexture)(GLenum); // TEXTURE0..TEXTURE1...
void (__stdcall *glCompressedTexImage2D)(GLenum, int, GLenum, GLsizei,
                                         GLsizei, int, GLsizei, void *);
void (__stdcall *glCompressedTexSubImage2D)(GLenum, int, int, int, GLsizei,
                                            GLsizei, GLenum, GLsizei,
                                            void *);
void (__stdcall *glGenerateMipmap)(GLenum);
#endif
@
\fimcodigo

Vamos agora inizializar estas 17 funções para que possam ser usadas:

\iniciocodigo
@<Windows: Configurar OpenGL@>+=
glActiveTexture = (void (__stdcall *)(GLenum)) load_function("glActiveTexture");
if(glActiveTexture == NULL) return false;
glCompressedTexImage2D = (void (__stdcall *)(GLenum, int, GLenum, GLsizei,
                            GLsizei, int, GLsizei, void *))
                                  load_function("glCompressedTexImage2D");
if(glCompressedTexImage2D == NULL) return false;
glCompressedTexSubImage2D = (void (__stdcall *)(GLenum, int, int, int, GLsizei,
                              GLsizei, GLenum, GLsizei, void *))
                                   load_function("glCompressedTexSubImage2D");
if(glCompressedTexSubImage2D == NULL) return false;
glGenerateMipmap = (void (__stdcall *)(GLenum)) load_function("glGenerateMipmap");
if(glGenerateMipmap == NULL) return false;
@
\fimcodigo

As funções acima também requerem que as seguintes macros sejam
definidas para serem usadas nas enumerações:

\iniciocodigo
@<Define Macros@>+=
#if defined(_WIN32)
#define GL_RGB                              0x1907
#define GL_RGBA                             0x1908
#define GL_ALPHA                            0x1906
#define GL_TEXTURE0                         0x84C0
#define GL_TEXTURE1                         0x84C1
#define GL_TEXTURE2                         0x84C2
#define GL_TEXTURE3                         0x84C3
#define GL_TEXTURE4                         0x84C4
#define GL_TEXTURE5                         0x84C5
#define GL_TEXTURE6                         0x84C6
#define GL_TEXTURE7                         0x84C7
#define GL_TEXTURE8                         0x84C8
#define GL_TEXTURE9                         0x84C9
#define GL_TEXTURE10                        0x84CA
#define GL_TEXTURE11                        0x84CB
#define GL_TEXTURE12                        0x84CC
#define GL_TEXTURE13                        0x84CD
#define GL_TEXTURE14                        0x84CE
#define GL_TEXTURE15                        0x84CF
#define GL_TEXTURE16                        0x84D0
#define GL_TEXTURE17                        0x84D1
#define GL_TEXTURE18                        0x84D2
#define GL_TEXTURE19                        0x84D3
#define GL_TEXTURE20                        0x84D4
#define GL_TEXTURE21                        0x84D5
#define GL_TEXTURE22                        0x84D6
#define GL_TEXTURE23                        0x84D7
#define GL_TEXTURE24                        0x84D8
#define GL_TEXTURE25                        0x84D9
#define GL_TEXTURE26                        0x84DA
#define GL_TEXTURE27                        0x84DB
#define GL_TEXTURE28                        0x84DC
#define GL_TEXTURE29                        0x84DD
#define GL_TEXTURE30                        0x84DE
#define GL_TEXTURE31                        0x84DF
#define GL_LUMINANCE                        0x1909
#define GL_TEXTURE_2D                       0x0DE1
#define GL_UNSIGNED_BYTE                    0x1401
#define GL_TEXTURE_WRAP_S                   0x2802
#define GL_TEXTURE_WRAP_T                   0x2803
#define GL_LUMINANCE_ALPHA                  0x190A
#define GL_TEXTURE_MAG_FILTER               0x2800
#define GL_TEXTURE_MIN_FILTER               0x2801
#define GL_UNSIGNED_SHORT_5_6_5             0x8363
#define GL_UNSIGNED_SHORT_4_4_4_4           0x8033
#define GL_UNSIGNED_SHORT_5_5_5_1           0x8034
#define GL_MAX_TEXTURE_IMAGE_UNITS          0x8872
#define GL_TEXTURE_CUBE_MAP_POSITIVE_X      0x8515
#define GL_TEXTURE_CUBE_MAP_POSITIVE_Y      0x8517
#define GL_TEXTURE_CUBE_MAP_POSITIVE_Z      0x8519
#define GL_TEXTURE_CUBE_MAP_NEGATIVE_X      0x8516
#define GL_TEXTURE_CUBE_MAP_NEGATIVE_Y      0x8518
#define GL_TEXTURE_CUBE_MAP_NEGATIVE_Z      0x851A
#define GL_MAX_VERTEX_TEXTURE_IMAGE_UNITS   0x8B4C
#define GL_MAX_COMBINED_TEXTURE_IMAGE_UNITS 0x8B4D
#endif
@
\fimcodigo

As funções OpenGL ES por fragmento são uma série de testes e operações
que podem modificar a cor de um pixel antes que ele seja efetivamente
desenhado em uma superfíce. Geralmente. se um dos testes falhar, o
pixel será descarado. As funções relacionadas a isso são:

\iniciocodigo
@<Declarações de Janela@>+=
#if defined(_WIN32)
extern void (__stdcall *glSampleCoverage)(GLclampf, bool);
extern void (__stdcall *glStencilFuncSeparate)(GLenum, GLenum, int, unsigned int);
extern void (__stdcall *glStencilOpSeparate)(GLenum, GLenum, GLenum, GLenum);
extern void (__stdcall *glBlendEquation)(GLenum);
extern void (__stdcall *glBlendEquationSeparate)(GLenum, GLenum);
extern void (__stdcall *glBlendFuncSeparate)(GLenum, GLenum);
extern void (__stdcall *glBlendColor)(GLclampf, GLclampf, GLclampf, GLclampf);
#endif
@
\fimcodigo

E a declaração destas funções no \monoespaco{window.c}:

\iniciocodigo
@<Variáveis Globais@>+=
#if defined(_WIN32)
void (__stdcall *glSampleCoverage)(GLclampf, bool);
void (__stdcall *glStencilFuncSeparate)(GLenum, GLenum, int, unsigned int);
void (__stdcall *glStencilOpSeparate)(GLenum, GLenum, GLenum, GLenum);
void (__stdcall *glBlendEquation)(GLenum);
void (__stdcall *glBlendEquationSeparate)(GLenum, GLenum);
void (__stdcall *glBlendFuncSeparate)(GLenum, GLenum);
void (__stdcall *glBlendColor)(GLclampf, GLclampf, GLclampf, GLclampf);
#endif
@
\fimcodigo

E a inicialização:

\iniciocodigo
@<Windows: Configurar OpenGL@>+=
glSampleCoverage = (void (__stdcall *)(GLclampf, bool))
                        load_function("glSampleCoverage");
if(glSampleCoverage == NULL) return false;
glStencilFuncSeparate = (void (__stdcall *)(GLenum, GLenum, int, unsigned int))
                           load_function("glStencilFuncSeparate");
if(glStencilFuncSeparate == NULL) return false;
glStencilOpSeparate = (void (__stdcall *)(GLenum, GLenum, GLenum, GLenum))
                         load_function("glStencilOpSeparate");
if(glStencilOpSeparate == NULL) return false;
glBlendEquation = (void (__stdcall *)(GLenum)) load_function("glBlendEquation");
if(glBlendEquation == NULL) return false;
glBlendEquationSeparate = (void (__stdcall *)(GLenum, GLenum))
                            load_function("glBlendEquationSeparate");
if(glBlendEquationSeparate == NULL) return false;
glBlendFuncSeparate = (void (__stdcall *)(GLenum, GLenum))
                        load_function("glBlendFuncSeparate");
if(glBlendFuncSeparate == NULL) return false;
glBlendColor = (void (__stdcall *)(GLclampf, GLclampf, GLclampf, GLclampf))
                   load_function("glBlendColor");
if(glBlendColor == NULL) return false;
@
\fimcodigo


O uso das funções acima requer também a definição das seguintes macros
a serem usadas como enumerações:

\iniciocodigo
@<Define Macros@>+=
#if defined(_WIN32)
#define GL_ONE                      1
#define GL_ZERO                     0
#define GL_LESS                     0x0201
#define GL_INCR                     0x1E02
#define GL_DECR                     0x1E03
#define GL_KEEP                     0x1E00
#define GL_BACK                     0x0405
#define GL_FRONT                    0x0404
#define GL_EQUAL                    0x0202
#define GL_NEVER                    0x0200
#define GL_ALWAYS                   0x0207
#define GL_LEQUAL                   0x0203
#define GL_GEQUAL                   0x0206
#define GL_INVERT                   0x150A
#define GL_REPLACE                  0x1E01
#define GL_GREATER                  0x0204
#define GL_NOTEQUAL                 0x0205
#define GL_FUNC_ADD                 0x8006
#define GL_INCR_WRAP                0x8507
#define GL_DECR_WRAP                0x8508
#define GL_SRC_ALPHA                0x0302
#define GL_DST_ALPHA                0x0304
#define GL_SRC_COLOR                0x0300
#define GL_DST_COLOR                0x0306
#define GL_FUNC_SUBTRACT            0x800A
#define GL_FRONT_AND_BACK           0x0408
#define GL_CONSTANT_COLOR           0x8001
#define GL_CONSTANT_ALPHA           0x8003
#define GL_SRC_ALPHA_SATURATE	    0x0308
#define GL_ONE_MINUS_SRC_COLOR      0x0301
#define GL_ONE_MINUS_DST_COLOR      0x0307
#define GL_ONE_MINUS_SRC_ALPHA      0x0303
#define GL_ONE_MINUS_DST_ALPHA      0x0305
#define GL_FUNC_REVERSE_SUBTRACT    0x800B
#define GL_ONE_MINUS_CONSTANT_COLOR 0x8002
#define GL_ONE_MINUS_CONSTANT_ALPHA 0x8004
#endif
@
\fimcodigo

Com relação à funções que afetam todo o ``framebuffer'', temos que
preparar estas 2 funções:

\iniciocodigo
@<Declarações de Janela@>+=
#if defined(_WIN32)
extern void (__stdcall *glStencilMaskSeparate)(GLenum, unsigned int);
extern void (__stdcall *glClearDepthf)(GLclampf);
#endif
@
\fimcodigo

A declaração delas em \monoespaco{window.c}:

\iniciocodigo
@<Variáveis Globais@>+=
#if defined(_WIN32)
void (__stdcall *glStencilMaskSeparate)(GLenum, unsigned int);
void (__stdcall *glClearDepthf)(GLclampf);
#endif
@
\fimcodigo

E a inicialização:

\iniciocodigo
@<Windows: Configurar OpenGL@>+=
glStencilMaskSeparate = (void (__stdcall *)(GLenum, unsigned int))
                            load_function("glStencilMaskSeparate");
if(glStencilMaskSeparate == NULL) return false;
glClearDepthf = (void (__stdcall *)(GLclampf)) load_function("glClearDepthf");
if(glClearDepthf == NULL) return false;
@
\fimcodigo

O último conjunto de funções faltantes são as funções que gerenciam
objetos de framebufer:

\iniciocodigo
@<Declarações de Janela@>+=
#if defined(_WIN32)
extern void (__stdcall *glBindFramefuffer)(GLenum, unsigned int);
extern void (__stdcall *glDeleteFramebuffers)(GLsizei, unsigned int *);
extern void (__stdcall *glGenFramebuffers)(GLsizei, unsigned int *);
extern void (__stdcall *glBindRenderbuffer)(GLenum, unsigned int);
extern void (__stdcall *glDeleteRenderbuffers)(GLsizei, const unsigned int *);
extern void (__stdcall *glGenRenderbuffers)(GLsizei, unsigned int *);
extern void (__stdcall *glRenderbufferStorage)(GLenum, GLenum, GLsizei, GLsizei);
extern void (__stdcall *glFramebufferRenderbuffer)(GLenum, GLenum, GLenum,
                                                   unsigned int);
extern void (__stdcall *glFramebufferTexture2D)(GLenum, GLenum, GLenum,
                                                unsigned int, int);
extern void (__stdcall *glCheckFramebufferStatus)(GLenum);
extern boolean (__stdcall *glIsFrabuffer)(unsigned int);
extern void (__stdcall *glGetFramebufferAttachmentParameteriv)(GLenum, GLenum,
                                                               GLenum, int *);
extern boolean (__stdcall *glIsRenderbuffer)(unsigned int);
extern void (__stdcall *glGetRenderbufferParameteriv)(GLenum, GLenum, int *);
#endif
@
\fimcodigo

A declaração destas 14 funções em \monoespaco{window.c}:

\iniciocodigo
@<Variáveis Globais@>+=
#if defined(_WIN32)
void (__stdcall *glBindFramefuffer)(GLenum, unsigned int);
void (__stdcall *glDeleteFramebuffers)(GLsizei, unsigned int *);
void (__stdcall *glGenFramebuffers)(GLsizei, unsigned int *);
void (__stdcall *glBindRenderbuffer)(GLenum, unsigned int);
void (__stdcall *glDeleteRenderbuffers)(GLsizei, const unsigned int *);
void (__stdcall *glGenRenderbuffers)(GLsizei, unsigned int *);
void (__stdcall *glRenderbufferStorage)(GLenum, GLenum, GLsizei, GLsizei);
void (__stdcall *glFramebufferRenderbuffer)(GLenum, GLenum, GLenum,
                                            unsigned int);
void (__stdcall *glFramebufferTexture2D)(GLenum, GLenum, GLenum,
                                         unsigned int, int);
void (__stdcall *glCheckFramebufferStatus)(GLenum);
boolean (__stdcall *glIsFrabuffer)(unsigned int);
void (__stdcall *glGetFramebufferAttachmentParameteriv)(GLenum, GLenum,
                                                        GLenum, int *);
boolean (__stdcall *glIsRenderbuffer)(unsigned int);
void (__stdcall *glGetRenderbufferParameteriv)(GLenum, GLenum, int *);
#endif
@
\fimcodigo

E a inicialização:

\iniciocodigo
@<Windows: Configurar OpenGL@>+=
glBindFramefuffer = (void (__stdcall *)(GLenum, unsigned int))
                       load_function("glBindFramefuffer");
if(glBindFramefuffer == NULL) return false;
glDeleteFramebuffers = (void (__stdcall *)(GLsizei, unsigned int *))
                           load_function("glDeleteFramebuffers");
if(glDeleteFramebuffers == NULL) return false;
glGenFramebuffers = (void (__stdcall *)(GLsizei, unsigned int *))
                       load_function("glGenFramebuffers");
if(glGenFramebuffers == NULL) return false;
glBindRenderbuffer = (void (__stdcall *)(GLenum, unsigned int))
                         load_function("glBindRenderbuffer");
if(glBindRenderbuffer == NULL) return false;
glDeleteRenderbuffers = (void (__stdcall *)(GLsizei, const unsigned int *))
                           load_function("glDeleteRenderbuffers");
if(glDeleteRenderbuffers == NULL) return false;
glGenRenderbuffers = (void (__stdcall *)(GLsizei, unsigned int *))
                         load_function("glGenRenderbuffers");
if(glGenRenderbuffers == NULL) return false;
glRenderbufferStorage = (void (__stdcall *)(GLenum, GLenum, GLsizei, GLsizei))
                           load_function("glRenderbufferStorage");
if(glRenderbufferStorage == NULL) return false;
glFramebufferRenderbuffer = (void (__stdcall *)(GLenum, GLenum, GLenum,
                                                unsigned int))
                              load_function("glFramebufferRenderbuffer");
if(glFramebufferRenderbuffer == NULL) return false;
glFramebufferTexture2D = (void (__stdcall *)(GLenum, GLenum, GLenum,
                                             unsigned int, int))
                            load_function("glFramebufferTexture2D");
if(glFramebufferTexture2D == NULL) return false;
glCheckFramebufferStatus = (void (__stdcall *)(GLenum))
                               load_function("glCheckFramebufferStatus");
if(glCheckFramebufferStatus == NULL) return false;
glIsFrabuffer = (boolean (__stdcall *)(unsigned int))
                   load_function("glIsFrabuffer");
if(glIsFrabuffer == NULL) return false;
glGetFramebufferAttachmentParameteriv = (void (__stdcall *)(GLenum, GLenum,
                                                            GLenum, int *))
                        load_function("glGetFramebufferAttachmentParameteriv");
if(glGetFramebufferAttachmentParameteriv == NULL) return false;
glIsRenderbuffer = (boolean (__stdcall *)(unsigned int))
                       load_function("glIsRenderbuffer");
if(glIsRenderbuffer == NULL) return false;
glGetRenderbufferParameteriv = (void (__stdcall *)(GLenum, GLenum, int *))
                                  load_function("glGetRenderbufferParameteriv");
if(glGetRenderbufferParameteriv == NULL) return false;
@
\fimcodigo

E as macros necessárias para o uso destas funções:

\iniciocodigo
@<Define Macros@>+=
#if defined(_WIN32)
#define GL_RGBA4                                        0x8056
#define GL_RGB565                                       0x8D62
#define GL_RGB5_A1                                      0x8057
#define GL_FRAMEBUFFER                                  0x8D40
#define GL_RENDERBUFFER                                 0x8D41
#define GL_STENCIL_INDEX8                               0x8D48
#define GL_DEPTH_ATTACHMENT                             0x8D00
#define GL_DEPTH_COMPONENT16                            0x81A5
#define GL_COLOR_ATTACHMENT0                            0x8CE0
#define GL_STENCIL_ATTACHMENT                           0x8D20
#define GL_RENDERBUFFER_WIDTH                           0x8D42
#define GL_RENDERBUFFER_HEIGHT                          0x8D43
#define GL_FRAMEBUFFER_COMPLETE                         0x8CD5
#define GL_RENDERBUFFER_RED_SIZE                        0x8D50
#define GL_RENDERBUFFER_BLUE_SIZE                       0x8D52
#define GL_RENDERBUFFER_GREEN_SIZE                      0x8D51
#define GL_RENDERBUFFER_ALPHA_SIZE                      0x8D53
#define GL_RENDERBUFFER_DEPTH_SIZE                      0x8D54
#define GL_RENDERBUFFER_STENCIL_SIZE                    0x8D55
#define GL_RENDERBUFFER_INTERNAL_FORMAT                 0x8D44
#define GL_FRAMEBUFFER_ATTACHMENT_OBJECT_TYPE           0x8CD0
#define GL_FRAMEBUFFER_ATTACHMENT_OBJECT_NAME           0x8CD1
#define GL_FRAMEBUFFER_ATTACHMENT_TEXTURE_LEVEL         0x8CD2
#define GL_FRAMEBUFFER_ATTACHMENT_TEXTURE_CUBE_MAP_FACE 0x8CD3
#endif
@
\fimcodigo



\subsubsecao{2.3.4. Escolhendo a Versão do OpenGL}

No código definido antes, em todos os ambientes, nós escolhemos como
versão do OpenGL aquilo que estivesse definido nas macros:

1) \monoespaco{W\_WINDOW\_OPENGL\_MAJOR\_VERSION}

2) \monoespaco{W\_WINDOW\_OPENGL\_MINOR\_VERSION}

Vamos agora definir qual será a versão padrão caso o usuário não
personalize estas macros.

Caso estejamos rodando o X11, estamos usango o EGL para criar um
contexto OpenGL ES. Por padrão iremos pedir então para criar um
contexto OpenGL ES 3.0. Isso fará com que todas as funções e recursos
do OpenGL ES 2.0 fiquem definidos, além de mais algumas coisas novas
da versão 3.0.

Se estivermos rodando em navegador de Internet por meio de Web
Assembly, então estaremos usando WebGL. A versão equivalente ao Open
GL ES 3.0 é o WebGL 2. Enquanto o primeiro WebGL seria equivalente ao
OpenGLES 2.0. Neste caso queremos a versão 2.

Já no Windows, nós não temos garantia de que é possível criar um
contexto OpenGL ES, pois isso depende do hardware em que estamos
rodando. Sendo assim, pedimos a criação de um contexto OpenGL 4.1, já
que é a versão mais garantida de funcionar que contém em sua API as
funções mais modernas do OpenGL Es.

Sendo assim, a nossa macro que define a versão OpenGL é:

\iniciocodigo
@<Cabeçalhos@>+=
#if defined(_WIN32) && !defined(W_WINDOW_OPENGL_MAJOR_VERSION)
#define W_WINDOW_OPENGL_MAJOR_VERSION 4
#define W_WINDOW_OPENGL_MINOR_VERSION 1
#elif defined(__EMSCRIPTEN__) && !defined(W_WINDOW_OPENGL_MAJOR_VERSION)
#define W_WINDOW_OPENGL_MAJOR_VERSION 2
#define W_WINDOW_OPENGL_MINOR_VERSION 0
#elif !defined(W_WINDOW_OPENGL_MAJOR_VERSION)
#define W_WINDOW_OPENGL_MAJOR_VERSION 3
#define W_WINDOW_OPENGL_MINOR_VERSION 0
#endif
@
\fimcodigo

\subsecao{2.4. Fechando uma Janela}

Fechar um janela faz a janela desaparecer e finaliza qualquer coisa
que tenha sido inicializada durante a criação da janela. O que inclui
o contexto OpenGL.

\subsubsecao{2.4.1. Closing a Window on X11}

Fechar uma janela no X11 significa invocar a função do X que pede ao
servidor para que a janela seja fechada. E além disso, fechar a
conexão com o servidor. Isso é feito chamando
respectivamente \monoespaco{XDestroyWindow}
e \monoespaco{XCloseDisplay}. Também fazemos uma checagem para ver se
realmente existe uma janela a serfechada e destruída.

\iniciocodigo
@<Funções da API@>+=
#if !defined(_WIN32) && !defined(__EMSCRIPTEN__)
bool _Wdestroy_window(void){
  if(already_have_window == false)
    return false;
  eglMakeCurrent(egl_display, EGL_NO_SURFACE, EGL_NO_SURFACE,
                 EGL_NO_CONTEXT );
  eglDestroySurface(egl_display, egl_window);
  eglDestroyContext(egl_display, egl_context);
  eglTerminate(egl_display);
  XDestroyWindow(display, window);
  XCloseDisplay(display);
  display = NULL;
  already_have_window = false;
  return true;
}
#endif
@
\fimcodigo

\subsubsecao{2.4.2. Fechando uma Janela em Web Assembly}

Fechar uma janela quando executando dentro de um navegador de Internet
graças ao Web Assembly significa finalizar todas as estruturas SDL e
esconder o canvas onde estávamos desenhando. Fazemos isso com a
seguinte função:

\iniciocodigo
@<Funções da API@>+=
#if defined(__EMSCRIPTEN__)
bool _Wdestroy_window(void){
  if(already_have_window == false)
    return false;
  SDL_FreeSurface(window);
  EM_ASM(
    var el = document.getElementById("canvas");
    el.style.display = "none";
    document.exitFullscreen();
  );
  already_have_window = false;
  return true;
}
#endif
@
\fimcodigo

\subsecao{2.4.3. Fechando uma Janela no Windows}

Fechar a janela no Windows significa chamar a  função que irá
encerrá-la:

\iniciocodigo
@<Funções da API@>+=
#if defined(_WIN32)
bool _Wdestroy_window(void){
  if(already_have_window == false)
    return false;
  wglMakeCurrent(NULL, NULL);
  wglDeleteContext(wgl_context);
  DestroyWindow(window);
  already_have_window = false;
  return true;
}
#endif
@
\fimcodigo

\subsecao{2.5. Renderizando a Janela}

Em um programa gráfico, geralmente temos um laço principal e dentro
deste laço, chamamos alguma função para desenhar na tela. Isso
significa pedir para que a janela seja atualizada depois de realizar
chamadas OpenGL para desenhar. O modo de fazer isso depende do nosso
ambiente e API.

\subsecao{2.5.1. Renderizando a Janela no X} 

No X, como pedimos para o EGL criar uma janela e por padrão ele cria
uma janela com buffer duplo, para renderizarmos os comandos OpenGL
após eles terem sido feitos, devemos fazer com que o buffer traseiro,
onde estávamos desenhando, se torne o buffer dianteiro que é exibido
na tela e vice-versa:

\iniciocodigo
@<Funções da API@>+=
#if !defined(_WIN32) && !defined(__EMSCRIPTEN__)
bool _Wrender_window(void){
  return eglSwapBuffers(egl_display, egl_window);
}
#endif
@
\fimcodigo

\subsecao{2.5.2. Renderizando a Janela no Emscripten} 

A função \monoespaco{emscripten\_sleep} seria a que realiza a
atualização de uma janela quando estamos em um laço
principal. Contudo, usá-la é uma má prática. No Web Assembly,
programas nunca devem rodar em um laço infinito, mas ao invés disso
uma função deve ser registrada para rodar várias vezes seguidas sem
parar ao invés de ser feito um laço infinito como é mais comum. Quando
registramos uma função desta forma, não é necessário usar nenhum
comando adicioinal para que a janela seja atualizada, isso é feito
automaticamente. Por isso, na nssa função abaixo, a única coisa que
fazemos é chamar uma função inócua apenas para garantir que enviamos
todos os comandos OpenGL. Mas não usamos nada para pedir para a tela
ser atualizada, já que assumimos que o programa não irá seguir a
má-prática:

\iniciocodigo
@<Funções da API@>+=
#if defined(__EMSCRIPTEN__)
bool _Wrender_window(void){
  glFlush();
  return true;
}
#endif
@
\fimcodigo

\subsecao{2.5.3. Renderizando a Janela no Windows}

No Windows, a função WGL responsável por trocar os buffers da janela e
assim tornar visíveis os desenhos feitos é
chamada \monoespaco{wglSwapLayerBuffers}:

\iniciocodigo
@<Funções da API@>+=
#if defined(_WIN32)
bool _Wrender_window(void){
  return wglSwapLayerBuffers(device_context, WGL_SWAP_MAIN_PLANE);
}
#endif
@
\fimcodigo

\subsecao{2.6. Obtendo o Tamanho da Janela}

Agora iremos definir a função que armazena nos ponteirso passados como
argumento o tamanho em pixels da nossa janela.

\subsubsecao{2.6.1. Obtendo o Tamanho da Janela no X}

No X11, a API do Xlib nos fornece a função \monoespaco{XGetGeometry}
que fornece informação sobre uma janela ou pixmap (imagem). A função
retorna uma grande quantidade de informação em diferentes
ponteiros. Mas a informação que importa para nós é apenas a altura e
largura da janela:

\iniciocodigo
@<Funções da API@>+=
#if !defined(_WIN32) && !defined(__EMSCRIPTEN__)
bool _Wget_window_size(int *width, int *height){
  Window root_window;
  int x, y;
  unsigned int border, depth;
  if(!already_have_window || display == NULL){
    *width = 0;
    *height = 0;
    return false;
  }
  XGetGeometry(display, window, &root_window, &x, &y,
               (unsigned int *) width, (unsigned int *) height, &border, &depth);
  window_size_y = *height;
  return true;
}
#endif
@
\fimcodigo

\subsubsecao{2.6.1. Obtendo o Tamanho da Janela no Web Assembly}

Em um navegador web, nós não temos uma janela, mas um ``canvas''
HTML. Por convenção, o canvas que usamos como janela tem o ID
``cancas''. Se queremos saber o seu tamanho, precisamos usar
Javascript para obter o elemento e em seguida ler sua altura e
largura.

\iniciocodigo
@<Funções da API@>+=
#if defined(__EMSCRIPTEN__)
bool _Wget_window_size(int *width, int *height){
  if(!already_have_window){
    *width = 0;
    *height = 0;
    return false;
  }
  *width = EM_ASM_INT({
    return document.getElementById("canvas").clientWidth;
  });
  *height = EM_ASM_INT({
    return document.getElementById("canvas").clientHeight;
  });
  window_size_y = *height;
  if(*width > 0 && *height > 0)
    return true;
  else{
    *width = 0;
    *height = 0;
    return false;
  }
}
#endif
@
\fimcodigo

\subsubsecao{2.6.3. Obtendo o Tamanho da Janela no WIndows}

No Windows nós usamos uma chamada a \monoespaco{GetWindowRect} para
obter o tamanho da janela. o qual é armazenado em uma
estrutura \monoespaco{RECT}:

\iniciocodigo
@<Funções da API@>+=
#if defined(_WIN32)
bool _Wget_window_size(int *width, int *height){
  BOOL ret;
  RECT rectangle;
  ret = GetWindowRect(window, &rectangle);
  if(ret){
    *width = rectangle.right - rectangle.left;
    *height = rectangle.bottom - rectangle.top;
    window_size_y = *height;
    return true;
  }
  else{
    *width = 0;
    *height = 0;
    return false;
  }
}
#endif
@
\fimcodigo

\secao{3. Gerenciando Entrada}

Nesta seção trataremos a detecção de eventos de entrada de um
usuário. Como quando ele pressiona uma tecla ou move o mouse.

\subsecao{3.1. Definindo o Teclado e Mouse}

Quando nós temos uma janela, podemos detectar entrada gerada pelo
usuário quando tal janela tem foco. Podemos como entrada detectar
eventos feitos pelo mouse e teclado.

Para nós, um teclado será representado pela seguinte variável de
estrutura global:

\iniciocodigo
@<Declarações de Janela@>+=
extern struct __Wkeyboard{
  long key[W_KEYBOARD_SIZE + 1]; // Vetor de teclas: contador de tempo para elas
} _Wkeyboard;
@
\fimcodigo

Emanteremos como variável global uma única estrutura assim para
representar o teclado do usuário:

\iniciocodigo
@<Variáveis Globais@>=
struct __Wkeyboard _Wkeyboard;
@
\fimcodigo


No vetor de teclas acima, nós reservamos uma posição para cada tecla
diferente com o propósito de contar o tempo que leva para serem
pressionadas e soltas. O significado do número armazenado em cada
posição depende de seu valor. As opções são:

1) Se armazenamos zero, isso significa que a tecla específica
associada à tal posição não está sendo pressionada.

2) Se armazenamos 1, isso significa que a tecla começou a ser
pressionada agora.

3) Se armazenamos um número positivo maior que 1, isso significa que a
tecla está sendo pressionada, ainda não foi solta e o número
representa a quantas unidades de tempo essa tecla está sendo
pressionada.

4) Se armazenamos um número negativo, isso significa que a tecla foi
solta agora. O oposto do número armazenado representa por quantas
unidades de tempo a tecla foi pressionada antes de ser solta.

O número de teclas acompanhadas depende do sistema operacional e
ambiente. A macro \monoespaco{W\_KEYBOARD\_SIZE} representa o número
de teclas que suportamos. A estrutura do teclado possui também uma
posição adicional para a qual será mapeada qualquer tecla
desconhecida. A última posição do vetor nunca armazenará nenhum valor
diferente de zero.

O mouse será representado pela seguinte estrutura:

\iniciocodigo
@<Declarações de Janela@>+=
extern struct __Wmouse{
  long button[W_MOUSE_SIZE];
  int x, y, dx, dy, ddx, ddy;
} _Wmouse;
@
\fimcodigo

E manteremos um único exemplar desta estrutura para representar o
mouse do usuário:

\iniciocodigo
@<Variáveis Globais@>+=
struct __Wmouse _Wmouse;
@
\fimcodigo

Os botões do mouse seguem as mesmas convenções das teclas de um
teclado. Nós podemos detectar se elas estão sendo pressionadas, por
quanto tempo, se elas são soltas e por quanto tempo foram pressionadas
antes de serem soltas.

As variáveis \monoespaco{x} e \monoespaco{y} representam a posição do
ponteiro do mouse em pixels de nossa janela. As
variáveis \monoespaco{dx} e \monoespaco{dy} representam a velocidade
do ponteiro em pixels por unidade de tempo. As
variáveis \monoespaco{ddx} e \monoespaco{ddy} representam a aceleração
do ponteiro em pixels por unidade de tempo ao quadrado.

Dadas estas estruturas, precisamos chamar a seguinte função
periodicamente para atualizá-las:

\iniciocodigo
@<Funções da API@>+=
void _Wget_window_input(unsigned long current_time){
  if(already_have_window == false)
    return;
  @<Antes de Obter Eventos da Janela@>
  @<Obter Eventos da Janela@>
  @<Depois de Obter Eventos da Janela@>
}
@
\fimcodigo

A função basicamente entra em um laço para ler todos os eventos que a
janela recebe. Dependendo do evento, ela atualiza de maneira adequada
o estado das estruturas da janela e mouse. Por exemplo, no X11, a
leitura de eventos acontece assim:

\iniciocodigo
@<Obter Eventos da Janela@>=
#if !defined(_WIN32) && !defined(__EMSCRIPTEN__)
XEvent event;
while(XPending(display)){
  XNextEvent(display, &event);
  @<X11: Tratar Eventos@>
}
#endif
@
\fimcodigo

Se estamos em ambiente Web Assembly, usamos a API SDL para ler os
eventos:

\iniciocodigo
@<Obter Eventos da Janela@>+=
#if defined(__EMSCRIPTEN__)
SDL_Event event;
while(SDL_PollEvent(&event)){
  @<Web Assembly: Tratar Eventos@>
}
#endif
@
\fimcodigo

No Windows, os eventos são chamados de ``mensagens''. E a forma de ler
eles é dada abaixo:

\iniciocodigo
@<Obter Eventos da Janela@>+=
#if defined(_WIN32)
MSG event;
while(PeekMessage(&event, window, WM_KEYFIRST, WM_KEYLAST, PM_REMOVE)){
  @<Windows: Tratar Eventos de Teclado@>
}
while(PeekMessage(&event, window, WM_MOUSEFIRST, WM_MOUSELAST, PM_REMOVE)){
  @<Windows: Tratar Eventos de Mouse@>
}
#endif
@
\fimcodigo

No caso do Windows, nós podemos filtrar o tipo de evento que queremos
tratar. Então separamos os eventos relacionados ao mouse e teclado,
tratando eles exclusivamente.

\subsecao{3.2. Lendo o Teclado}

Aqui definiremos como iremos monitorar o teclado em diferentes
ambientes para atualizar a estrutura de teclado que definimos.

Os eventos, em todos os ambientes, vao nos avisar se uma tecla é
pressionada ou solta. Mas eles não vao nos passar qualquer informação
sobre o tempo que elas ficaram pressionadas. Nós temos que armazenar e
cuidar desta informação. Para isso, usaremos um vetor que armamazena
teclas e o tempo no qual recebemos a informação de que elas foram
pressionadas ou soltas. No laço em que lemos os eventos, é esse vetor
de teclas que iremos atualizar:

\iniciocodigo
@<Variáveis Locais@>=
static struct{
  unsigned key; // Qual tecla foi pressionada?
  long time;    // Quando foi pressionada?
} pressed_keys[32];
static unsigned released_keys[32]; // Que teclas foram soltas?
@
\fimcodigo

Como parte da inicialização de nosso teclado, e também para reiniciar
o estado dele, devemos chamar sempre um código que percorre a lista e
deixa cada uma das posições marcada como tendo uma tecla de valor
zero. Iremos usar uma tecla de valor zero para marcar o fim da nossa
lista, exatamente como usa-se um caractere de valor zero para marcar o
fim de uma string. Além disso, quando queremos inicializar ou
reinicializar o teclado, também temos que percorrer o seu vetor de
teclas e ajustar todos para zero. O código de inicialização e
reinicialização do teclado é:

\iniciocodigo
@<Iniciando ou Reiniciando Teclado@>=
{
  int i;
  for(i = 0; i < 32; i ++){
    pressed_keys[i].key = 0;
    released_keys[i] = 0;
  }
  for(i = 0; i < W_KEYBOARD_SIZE + 1; i ++)
    _Wkeyboard.key[i] = 0;
}
@
\fimcodigo

\subsecao{3.2.1. Lendo o Teclado no X}

No X11, cada tecla de teclado diferente possui um códico único entre 8
e 255. Esse código representa a tecla física pressionada, mas não tem
relação com o símbolo específico associado à tal tecla. O símbolo
depende do mapeamento de teclado configurado. Para obter o símbolo
associado à tecla pressionada, é necessário traduzir o código para um
símbolo (\monoespaco{keycode} para \monoespaco{keysym}). De qualquer
forma, isso significa que teremos que acompanhar um número potencial
de quase 256 teclas diferentes.

\iniciocodigo
@<Define Macros@>+=
#if !defined(_WIN32) && !defined(__EMSCRIPTEN__)
#define W_KEYBOARD_SIZE 256
#endif
@
\fimcodigo

Para detectar se uma tecla foi pressionada, antes de adicioná-la à
lista de teclas pressionadas, usamos o seguinte código:

\iniciocodigo
@<X11: Tratar Eventos@>=
if(event.type == KeyPress){
  unsigned key = event.xkey.keycode;
  @<Adiciona 'key' à Lista de Teclas Pressionadas@>
}
@
\fimcodigo

Para detectar se uma tecla foi solta, antes de podermos removê-la da
lista de teclas pressionadas e adicionar à lista de teclas soltas,
usamos o código abaixo:

\iniciocodigo
@<X11: Tratar Eventos@>+=
if(event.type == KeyRelease){
  unsigned key = event.xkey.keycode;
  @<Remove 'key' da Lista de Teclas Pressionadas e Adiciona às Teclas Soltas@>
}
@
\fimcodigo

Embora o código acima trate corretamente os códigos de teclas, um
usuário também deve saber o que os códigos representam. Ser informado
que a tecla de código 0xff0d foi pressionada significa muito
pouco. Saber que a tecla ``Enter'' foi pressionada nos dá muito mais
informação. O usuário deve ser capaz de checar o conteúdo
de \monoespaco{\_Wkeyboard.key[W\_ENTER]} sem ter que saber qual o código
para a tecla no teclado que está sendo usado. Para isso, na
inicialização de teclado, devemos descobrir todos os símbolos que
suportamos no teclado para assim darmos a eles nomes adequados. Essa
informação depende de como o teclado está mapeado, então só podemos
obtê-la durante a execução de um programa. O código para descobrirmos
a posição de cada tecla que suportamos é:



\iniciocodigo
@<Inicialização de Teclado@>=
#if !defined(_WIN32) && !defined(__EMSCRIPTEN__)
{
  int i;
  for(i = 8; i < 256; i ++){
    unsigned long value = XkbKeycodeToKeysym(display, i, 0, 0);
    switch(value){
    case 0: break;
    case XK_Escape: W_ESC = i; break;
    case XK_BackSpace: W_BACKSPACE = i; break;
    case XK_Tab: W_TAB = i; break;
    case XK_Return: W_ENTER = i; break;
    case XK_Up:   W_UP   = i; break; case XK_Down:  W_DOWN =  i; break;
    case XK_Left: W_LEFT = i; break; case XK_RIGHT: W_RIGHT = i; break;
    case XK_0: W_0 = i; break;     case XK_1: W_1 = i; break;
    case XK_2: W_2 = i; break;     case XK_3: W_3 = i; break;
    case XK_4: W_4 = i; break;     case XK_5: W_5 = i; break;
    case XK_6: W_6 = i; break;     case XK_7: W_7 = i; break;
    case XK_8: W_8 = i; break;     case XK_9: W_9 = i; break;
    case XK_minus: W_MINUS = i; break;    case XK_plus: W_PLUS = i;
    case XK_F1: W_F1 = i; break;   case XK_F2: W_F2 = i; break;
    case XK_F3: W_F3 = i; break;   case XK_F4: W_F4 = i; break;
    case XK_F5: W_F5 = i; break;   case XK_F6: W_F6 = i; break;
    case XK_F7: W_F7 = i; break;   case XK_F8: W_F8 = i; break;
    case XK_F9: W_F9 = i; break;   case XK_F10: W_F10 = i; break;
    case XK_F11: W_F11 = i; break; case XK_F12: W_F12 = i; break;
    case XK_Shift_L: W_LEFT_SHIFT = i; break;
    case XK_Shift_R: W_RIGHT_SHIFT = i; break;
    case XK_Control_L: W_LEFT_CTRL = i; break;
    case XK_Control_R: W_RIGHT_CTRL = i; break;
    case XK_Alt_L: W_LEFT_ALT = i; break;
    case XK_Alt_R: W_RIGHT_ALT = i; break;
    case XK_space: W_SPACE = i; break;
    case XK_a: W_A = i; break;   case XK_b: W_B = i; break;
    case XK_c: W_C = i; break;   case XK_d: W_D = i; break;
    case XK_e: W_E = i; break;   case XK_f: W_F = i; break;
    case XK_g: W_G = i; break;   case XK_h: W_H = i; break;
    case XK_i: W_I = i; break;   case XK_j: W_J = i; break;
    case XK_k: W_K = i; break;   case XK_l: W_L = i; break;
    case XK_m: W_M = i; break;   case XK_n: W_N = i; break;
    case XK_o: W_O = i; break;   case XK_p: W_P = i; break;
    case XK_q: W_Q = i; break;   case XK_r: W_R = i; break;
    case XK_s: W_S = i; break;   case XK_t: W_T = i; break;
    case XK_u: W_U = i; break;   case XK_v: W_V = i; break;
    case XK_w: W_W = i; break;   case XK_x: W_X = i; break;
    case XK_y: W_Y = i; break;   case XK_z: W_Z = i; break;
    case XK_Insert: W_INSERT = i; break;
    case XK_Home: W_HOME = i; break;
    case XK_Page_Up: W_PAGE_UP = i; break;
    case XK_Delete: W_DELETE = i; break;
    case XK_End: W_END = i; break;
    case XK_Page_Down: W_PAGE_DOWN = i; break;
    default: break;
    }
  }
}
#endif
@
\fimcodigo

Todas estas variáveis que representam posições no nosso vetor de
teclas de teclado precisam ser declaradas:

\iniciocodigo
@<Declarações de Janela@>+=
extern int W_BACKSPACE, W_TAB, W_ENTER, W_UP, W_DOWN, W_LEFT, W_RIGHT, W_0, W_1,
           W_2, W_3, W_4, W_5, W_6, W_7, W_8, W_9, W_MINUS, W_PLUS, W_F1, W_F2,
           W_F3, W_F4, W_F5, W_F6, W_F7, W_F8, W_F9, W_F10, W_F11, W_F12,
           W_LEFT_SHIFT, W_RIGHT_SHIFT, W_LEFT_ALT, W_RIGHT_ALT, W_LEFT_CTRL,
           W_RIGHT_CTRL, W_SPACE, W_A, W_B, W_C, W_D, W_E, W_F, W_G, W_H, W_I,
           W_J, W_K, W_L, W_M, W_N, W_O, W_P, W_Q, W_R, W_S, W_T, W_U, W_V, W_W,
           X_X, W_Y, W_Z, W_INSERT, W_HOME, W_PAGE_UP, W_DELETE, W_END,
           W_PAGE_DOWN, W_ESC, W_ANY;
@
\fimcodigo

E elas são inicializadas como sendo o
valor \monoespaco{W\_KEYBOARD\_SIZE}, reservado para representar
teclas desconhecidas. Somente depois da inicialização do teclado as
variáveis que representam teclas reconhecidas são mapeadas para sua
verdadeira posição.

\iniciocodigo
@<Variáveis Globais@>+=
int W_BACKSPACE = W_KEYBOARD_SIZE, W_TAB = W_KEYBOARD_SIZE,
    W_ENTER = W_KEYBOARD_SIZE, W_UP = W_KEYBOARD_SIZE, W_DOWN = W_KEYBOARD_SIZE,
    W_LEFT = W_KEYBOARD_SIZE, W_RIGHT = W_KEYBOARD_SIZE, W_0 = W_KEYBOARD_SIZE,
    W_1 = W_KEYBOARD_SIZE, W_2 = W_KEYBOARD_SIZE, W_3 = W_KEYBOARD_SIZE,
    W_4 = W_KEYBOARD_SIZE, W_5 = W_KEYBOARD_SIZE, W_6 = W_KEYBOARD_SIZE,
    W_7 = W_KEYBOARD_SIZE, W_8 = W_KEYBOARD_SIZE, W_9 = W_KEYBOARD_SIZE,
    W_MINUS = W_KEYBOARD_SIZE, W_PLUS = W_KEYBOARD_SIZE, W_F1 = W_KEYBOARD_SIZE,
    W_F2 = W_KEYBOARD_SIZE, W_F3 = W_KEYBOARD_SIZE, W_F4 = W_KEYBOARD_SIZE,
    W_F5 = W_KEYBOARD_SIZE, W_F6 = W_KEYBOARD_SIZE, W_F7 = W_KEYBOARD_SIZE,
    W_F8 = W_KEYBOARD_SIZE, W_F9 = W_KEYBOARD_SIZE, W_F10 = W_KEYBOARD_SIZE,
    W_F11 = W_KEYBOARD_SIZE, W_F12 = W_KEYBOARD_SIZE,
    W_LEFT_SHIFT = W_KEYBOARD_SIZE, W_RIGHT_SHIFT = W_KEYBOARD_SIZE,
    W_LEFT_ALT = W_KEYBOARD_SIZE, W_RIGHT_ALT = W_KEYBOARD_SIZE,
    W_LEFT_CTRL = W_KEYBOARD_SIZE, W_RIGHT_CTRL = W_KEYBOARD_SIZE,
    W_SPACE = W_KEYBOARD_SIZE, W_A = W_KEYBOARD_SIZE, W_B = W_KEYBOARD_SIZE,
    W_C = W_KEYBOARD_SIZE, W_D = W_KEYBOARD_SIZE, W_E = W_KEYBOARD_SIZE,
    W_F = W_KEYBOARD_SIZE, W_G = W_KEYBOARD_SIZE, W_H = W_KEYBOARD_SIZE,
    W_I = W_KEYBOARD_SIZE, W_J = W_KEYBOARD_SIZE, W_K = W_KEYBOARD_SIZE,
    W_L = W_KEYBOARD_SIZE, W_M = W_KEYBOARD_SIZE, W_N = W_KEYBOARD_SIZE,
    W_O = W_KEYBOARD_SIZE, W_P = W_KEYBOARD_SIZE, W_Q = W_KEYBOARD_SIZE,
    W_R = W_KEYBOARD_SIZE, W_S = W_KEYBOARD_SIZE, W_T = W_KEYBOARD_SIZE,
    W_U = W_KEYBOARD_SIZE, W_V = W_KEYBOARD_SIZE, W_W = W_KEYBOARD_SIZE,
    X_X = W_KEYBOARD_SIZE, W_Y = W_KEYBOARD_SIZE, W_Z = W_KEYBOARD_SIZE,
    W_INSERT = W_KEYBOARD_SIZE, W_HOME = W_KEYBOARD_SIZE,
    W_PAGE_UP = W_KEYBOARD_SIZE, W_DELETE = W_KEYBOARD_SIZE,
    W_END = W_KEYBOARD_SIZE, W_PAGE_DOWN = W_KEYBOARD_SIZE,
    W_ESC = W_KEYBOARD_SIZE, W_ANY = 0;
@
\fimcodigo

\subsecao{3.2.2. Lendo o Teclado no Web Assembly}

O Emscripten fornece para nós a API do SDL para interagir com o
teclado. Nesta API, o número de teclas suportadas é definida pela
macro \monoespaco{SDL\_NUM\_SCANCODES}, que na épóca de escrita deste
texto corresponde a 512 teclas diferentes. Provavelmente o valor não
mudará no futuro próximo.

\iniciocodigo
@<Define Macros@>+=
#if defined(__EMSCRIPTEN__)
#define W_KEYBOARD_SIZE SDL_NUM_SCANCODES
#endif
@
\fimcodigo

No X11, vimos que há uma diferença entre a tecla física pressionada e
o síbolo que ela representa. A mesma diferença aparece aqui. A tecla
física no X é chamada de \monoespaco{Keycode} a aqui no SDL é chamada
de \monoespaco{Scancode}. O símbolo que ela representa no X era
chamado de \monoespaco{Keysym} e aqui é chamada
de \monoespaco{Scancode}. Sim, a nomenclatura é confusa pois
o \monoespaco{Scancode} significa coisas completamente diferentes nas
duas APIs. Usando a nomenclatura do SDL, é assim que detectamos uma
tecla sendo pressionada:

\iniciocodigo
@<Web Assembly: Tratar Eventos@>=
if(event.type == SDL_KEYDOWN){
  unsigned key = event.key.keysym.scancode;
  @<Adiciona 'key' à Lista de Teclas Pressionadas@>
}
@
\fimcodigo

Já para tratar o caso de estarmos soltando uma tecla, usamos o código
abaixo:

\iniciocodigo
@<Web Assembly: Tratar Eventos@>+=
if(event.type == SDL_KEYUP){
  unsigned key = event.key.keysym.scancode;
  @<Remove 'key' da Lista de Teclas Pressionadas e Adiciona às Teclas Soltas@>
}
@
\fimcodigo

O código acima é idêntico ao usado no X11, apenas foi necessário mudar
os nomes de variáveis seguindo a nomenclatura da estrutura do
SDL. Agora precisamos do código que inicializa corretamente o nome de
cada uma destas teclas:

\iniciocodigo
@<Inicialização de Teclado@>+=
#if defined(__EMSCRIPTEN__)
{
  int i;
  for(i = 0; i < W_KEYBOARD_SIZE; i ++){
    unsigned long value = SDL_GetKeyFromScancode(i);
    switch(value){
    case 0: break;
    case SDLK_ESCAPE: W_ESC = i; break;
    case SDLK_BACKSPACE: W_BACKSPACE = i; break;
    case SDLK_TAB: W_TAB = i; break;
    case SDLK_RETURN: W_ENTER = i; break;
    case SDLK_UP:   W_UP   = i; break; case SDLK_DOWN:  W_DOWN =  i; break;
    case SDLK_LEFT: W_LEFT = i; break; case SDLK_RIGHT: W_RIGHT = i; break;
    case SDLK_0: W_0 = i; break;     case SDLK_1: W_1 = i; break;
    case SDLK_2: W_2 = i; break;     case SDLK_3: W_3 = i; break;
    case SDLK_4: W_4 = i; break;     case SDLK_5: W_5 = i; break;
    case SDLK_6: W_6 = i; break;     case SDLK_7: W_7 = i; break;
    case SDLK_8: W_8 = i; break;     case SDLK_9: W_9 = i; break;
    case SDLK_MINUS: W_MINUS = i; break;    case SDLK_PLUS: W_PLUS = i;
    case SDLK_F1: W_F1 = i; break;   case SDLK_F2: W_F2 = i; break;
    case SDLK_F3: W_F3 = i; break;   case SDLK_F4: W_F4 = i; break;
    case SDLK_F5: W_F5 = i; break;   case SDLK_F6: W_F6 = i; break;
    case SDLK_F7: W_F7 = i; break;   case SDLK_F8: W_F8 = i; break;
    case SDLK_F9: W_F9 = i; break;   case SDLK_F10: W_F10 = i; break;
    case SDLK_F11: W_F11 = i; break; case SDLK_F12: W_F12 = i; break;
    case SDLK_LSHIFT: W_LEFT_SHIFT = i; break;
    case SDLK_RSHIFT: W_RIGHT_SHIFT = i; break;
    case SDLK_LCTRL: W_LEFT_CTRL = i; break;
    case SDLK_RCTRL: W_RIGHT_CTRL = i; break;
    case SDLK_LALT: W_LEFT_ALT = i; break;
    case SDLK_RALT: W_RIGHT_ALT = i; break;
    case SDLK_SPACE: W_SPACE = i; break;
    case SDLK_a: W_A = i; break;   case SDLK_b: W_B = i; break;
    case SDLK_c: W_C = i; break;   case SDLK_d: W_D = i; break;
    case SDLK_e: W_E = i; break;   case SDLK_f: W_F = i; break;
    case SDLK_g: W_G = i; break;   case SDLK_h: W_H = i; break;
    case SDLK_i: W_I = i; break;   case SDLK_j: W_J = i; break;
    case SDLK_k: W_K = i; break;   case SDLK_l: W_L = i; break;
    case SDLK_m: W_M = i; break;   case SDLK_n: W_N = i; break;
    case SDLK_o: W_O = i; break;   case SDLK_p: W_P = i; break;
    case SDLK_q: W_Q = i; break;   case SDLK_r: W_R = i; break;
    case SDLK_s: W_S = i; break;   case SDLK_t: W_T = i; break;
    case SDLK_u: W_U = i; break;   case SDLK_v: W_V = i; break;
    case SDLK_w: W_W = i; break;   case SDLK_x: W_X = i; break;
    case SDLK_y: W_Y = i; break;   case SDLK_z: W_Z = i; break;
    case SDLK_INSERT: W_INSERT = i; break;
    case SDLK_HOME: W_HOME = i; break;
    case SDLK_PAGEUP: W_PAGE_UP = i; break;
    case SDLK_DELETE: W_DELETE = i; break;
    case SDLK_END: W_END = i; break;
    case SDLK_PAGEDOWN: W_PAGE_DOWN = i; break;
    default: break;
    }
  }
}
#endif
@
\fimcodigo

\subsecao{3.2.3. Lendo o Teclado no WIndows}

No Windows as teclas físicas são chamadas pelo nome
de \monoespaco{Scancode} e o símbolo associado a ela é chamado de
``código de tecla virtural''. O número de diferentes ``scancodes'' é
sempre 256 e precisamos de uma posiçã a mais para representar ``qualquer
tecla'':

\iniciocodigo
@<Define Macros@>+=
#if defined(_WIN32)
#define W_KEYBOARD_SIZE 256
#endif
@
\fimcodigo

Detectar teclas sendo pressionadas é feito de maneira idêntica ao
código que já vimos para o X11 e SDL, apenas com modificações menores
para abarcar algumas diferenças, e também tendo que mudar o nome das
coisas, já que o Windows nomeia de maniera diferente as funções e
estruturas:

\iniciocodigo
@<Windows: Tratar Eventos de Teclado@>=
if(event.message == WM_KEYDOWN){
  unsigned key = (event.lParam & 0x00ff0000) >> 16;
  @<Adiciona 'key' à Lista de Teclas Pressionadas@>
}
@
\fimcodigo

Note que uma das diferenças aqui é que para extrair o ``scancode'' da
tecla física, precisamos fazer um pouco de manipulação de bits. Fora
isso, o código é o mesmo que já foi visto nas subsubseções anteriores
para X11 e SDL. Já o código que detecta se uma tecla foi solta é:

\iniciocodigo
@<Windows: Tratar Eventos de Teclado@>+=
if(event.message == WM_KEYUP){
  unsigned key = (event.lParam & 0x00ff0000) >> 16;
  @<Remove 'key' da Lista de Teclas Pressionadas e Adiciona às Teclas Soltas@>
}
@
\fimcodigo

E finalmente, inicializaos aqui os nomes das teclas para que eles
apontem para a posição correta no nosso vetor de teclas:

\iniciocodigo
@<Inicialização de Teclado@>+=
#if defined(_WIN32)
{
  int i;
  for(i = 0; i < W_KEYBOARD_SIZE; i ++){
    unsigned long value = MapVirtualKey(i, MAPVK_VSK_TO_VSC_EX);
    switch(value){
    case 0: break;
    case VK_ESCAPE: W_ESC = i; break; 
    case VK_BACK: W_BACKSPACE = i; break;
    case VK_TAB: W_TAB = i; break;
    case VK_RETURN: W_ENTER = i; break;
    case VK_UP:   W_UP   = i; break; case VK_DOWN:  W_DOWN =  i; break;
    case VK_LEFT: W_LEFT = i; break; case VK_RIGHT: W_RIGHT = i; break;
    case '0': W_0 = i; break;     case '1': W_1 = i; break;
    case '2': W_2 = i; break;     case '3': W_3 = i; break;
    case '4': W_4 = i; break;     case '5': W_5 = i; break;
    case '6': W_6 = i; break;     case '7': W_7 = i; break;
    case '8': W_8 = i; break;     case '9': W_9 = i; break;
    case VK_OEM__MINUS: W_MINUS = i; break;    case VK_OEM__PLUS: W_PLUS = i;
    case VK_F1: W_F1 = i; break;   case VK_F2: W_F2 = i; break;
    case VK_F3: W_F3 = i; break;   case VK_F4: W_F4 = i; break;
    case VK_F5: W_F5 = i; break;   case VK_F6: W_F6 = i; break;
    case VK_F7: W_F7 = i; break;   case VK_F8: W_F8 = i; break;
    case VK_F9: W_F9 = i; break;   case VK_F10: W_F10 = i; break;
    case VK_F11: W_F11 = i; break; case VK_F12: W_F12 = i; break;
    case VK_LSHIFT: W_LEFT_SHIFT = i; break;
    case VK_RSHIFT: W_RIGHT_SHIFT = i; break;
    case VK_LCONTROL: W_LEFT_CTRL = i; break;
    case VK_RCONTROL: W_RIGHT_CTRL = i; break;
    case VK_MENU: W_LEFT_ALT = i; break;
    case VK_RMENU: W_RIGHT_ALT = i; break;
    case VK_SPACE: W_SPACE = i; break;
    case 'A': W_A = i; break;   case 'B': W_B = i; break;
    case 'C': W_C = i; break;   case 'D': W_D = i; break;
    case 'E': W_E = i; break;   case 'F': W_F = i; break;
    case 'G': W_G = i; break;   case 'H': W_H = i; break;
    case 'I': W_I = i; break;   case 'J': W_J = i; break;
    case 'K': W_K = i; break;   case 'L': W_L = i; break;
    case 'M': W_M = i; break;   case 'N': W_N = i; break;
    case 'O': W_O = i; break;   case 'P': W_P = i; break;
    case 'Q': W_Q = i; break;   case 'R': W_R = i; break;
    case 'S': W_S = i; break;   case 'T': W_T = i; break;
    case 'U': W_U = i; break;   case 'V': W_V = i; break;
    case 'W': W_W = i; break;   case 'X': W_X = i; break;
    case 'Y': W_Y = i; break;   case 'Z': W_Z = i; break;
    case VK_INSERT: W_INSERT = i; break;
    case VK_HOME: W_HOME = i; break;
    case VK_PRIOR: W_PAGE_UP = i; break;
    case VK_DELETE: W_DELETE = i; break;
    case VK_END: W_END = i; break;
    case VK_NEXT: W_PAGE_DOWN = i; break;
    default: break;
    }
  }
}
#endif
@
\fimcodigo

\subsecao{3.2.4. Código Adicional para Suporte de Teclado}

Com o código que foi escrito até agora nós podemos detectar quando uma
nova tecla é pressionada ou solta. Tal código não é portável, depende
do Sistema Operacional. Mas nós ainda não escrevemos o código que
adiciona tal informação à nossa lista de teclas pressionadas e soltas
e nem o código que lê informações destas listas para atualizar a
estrutura de nosso teclado.

Primeiro de tudo, vamos definir como iremos adicionar uma nova tecla à
lista de teclas pressionadas. Lembre-se que a esta altura, em todos os
Sistemas Operacionais e ambientes, já escrevemos o código que armazena
a informação de qual tecla foi pressionada na
variável \monoespaco{key}. E lembre-se que nossa lista de teclas
pressionadas representa o fim da lista por uma tecla nula
armazenada. Sabendo disso, adicionar uma tecla na lista significa
iterar sobre ela até achar uma teclas nula e fazer a substituição:

\iniciocodigo
@<Adiciona 'key' à Lista de Teclas Pressionadas@>=
{
  int i;
  for(i = 0; i < 32; i ++){
    if(pressed_keys[i].key == key) // Já tava pressionada
      break;
    if(pressed_keys[i].key == 0){ // Começou a ser pressionada agora
      pressed_keys[i].key = key;
      pressed_keys[i].time = current_time;
      // Atualizando teclado:
      _Wkeyboard.key[key] = 1;
      break;
    }
  }
  if(i == 32) continue; // Ignoring: too many keypresses
}
@
\fimcodigo

Agora para removermos uma tecla da lista de teclas pressionadas e
adicioná-las à lista de teclas soltas, primeiro iteramos sobre s
teclas pressionadas até acharmos a que queremos remover. Em seguida
movemos todas as teclas das posições seguintes para uma posição
anterior, o que efetivamente apaga a tecla desejada da lista. Depois
disso, iteramos sobre a lista de teclas soltas até achar uma posição
vaga e inserimos nossa nova tecla ali:

\iniciocodigo
@<Remove 'key' da Lista de Teclas Pressionadas e Adiciona às Teclas Soltas@>=
{
  int i;
  long stored_time = -1;
  for(i = 0; i < 32; i ++){ // Removing from pressed keys
    if(pressed_keys[i].key == key){
      int j;
      stored_time = pressed_keys[i].time;
      for(j = i; j < 31; j ++){
        pressed_keys[j].key = pressed_keys[j + 1].key;
        pressed_keys[j].time = pressed_keys[j + 1].time;
        if(pressed_keys[j].key == 0)
          break;
      }
      pressed_keys[31].key = 0;
      break;
    }
  }
  for(i = 0; i < 32; i ++){ // Adding to released keys:
    if(released_keys[i] == 0)
      released_keys[i] = key;
  }
  if(i == 32) // Too many key releases, ignore and clean keyboard
    _Wkeyboard.key[key] = 0;
  else{
    // Updating keyboard struct:
    _Wkeyboard.key[key] = - (long) (current_time - stored_time);
    if(_Wkeyboard.key[key] == 0)
      _Wkeyboard.key[key] = -1; // Press/release in the same frame
  }
}
@
\fimcodigo

Mas como usar a informação da lista de teclas pressionadas e soltas
para atualizar nossa estrutura de teclado? Primeiro, a cada frame,
depois de atualizarmos a lista, devemos iterar sobre nossa lista de
teclas pressionadas. E para cada uma das teclas, devemos atualizar a
contagem de tempo que elas estão sendo pressionadas na nossa estrutura
de teclado:

\iniciocodigo
@<Depois de Obter Eventos da Janela@>=
{
  int i;
  // Update the information if any key is being pressed:
  _Wkeyboard.key[W_ANY] = (pressed_keys[0].key != 0);
  for(i = 0; i < 32; i ++){
    if(pressed_keys[i].key == 0)
      break;
    if(current_time > pressed_keys[i].key)
      _Wkeyboard.key[pressed_keys[i].key] = (current_time - pressed_keys[i].key);
  }
}
@
\fimcodigo

Agora o contador de tempo de teclas pressionadas vai atualizar a cada
frame, nao só ficar sempre no valor de 1.

Para quando a tecla é solta, nós já escrevemos código que torna
negativo o valor presente no contador de tempo. Mas não escrevemos o
código que faz o contador voltar a ser zero no frame seguinte a ele
ser solto. Para isso, antes de lermos a entrada da janela e
atualizarmos a lista de teclas pressionadas, devemos iterar sobre a
lista de teclas que foram soltas no último frame e para cada uma
delas ajustar o número de cada tecla para zero e esvaziar a lista de
teclas soltas:

\iniciocodigo
@<Antes de Obter Eventos da Janela@>=
{
  int i;
  for(i = 0; i < 32; i ++){
    if(released_keys[i] == 0)
      break;
    _Wkeyboard.key[released_keys[i]] = 0;
    released_keys[i] = 0;
  }
}
@
\fimcodigo

\subsecao{3.3. Lendo o Mouse}

Como com as teclas do teclado, também precisamos armazenar uma lista
de botões que estão sendo pressionados ou soltos. Mas o número de
botões é muito menor. Nossas listas irão apenas armazenar no máximo 4
botões. É pouco realista dar significado a ações que envolvem
pressionar mais do que quatro botões de teclado:

\iniciocodigo
@<Variáveis Locais@>+=
static struct{
  unsigned button; // Which button was pressed?
  long time;       // When it was pressed?
} pressed_buttons[4];
static unsigned released_buttons[4]; // Which buttons were released?
@
\fimcodigo

Mas ao contrário do teclado, um mouse pode se mover. E se queremos
saber não apenas a posição do mouse, mas sua velocidade e aceleração,
devemos armazenar como informação adicional a sua última
velocidade. Vamos usar estas variáveis para isso:

\iniciocodigo
@<Variáveis Locais@>+=
static int last_mouse_dx = 0, last_mouse_dy = 0;
@
\fimcodigo

Por fim, quando lemos pela primeira vez a posição do mouse, não temos
uma leitura anterior de sua posição e velocidae. Então não faz sentido
calcularmos neste momento nem a velocidade e nem a aceleração. Já
depois de passar o primeiro frame após a inicialização, temos a
posição anterior e a atual, o que nos permite obter um valor correto
para a velocidade. Mas ainda não podemos saber qual é a
aceleração. Somente no segundo frame após a inicialização que podemos
calcular todas estas informações. Até lá, é interessante manter como
zero valores que não temos como calcular. Por causa disso, vamos usar
uma variável adicional que indica se o mouse ainda está
inicializando. O valor 3 indica que não foi inicializada e não sabemos
nem a posição. O valor 2 indica que ele esta inicializando agora e não
temos como saber a velocidade. O valor 1 indica que nã há ainda como
obter aceleração. E o valor 0 indica que a inicialização terminou. A
variável que armazenará isso é:

\iniciocodigo
@<Variáveis Locais@>+=
static int mouse_initialization = 3;
@
\fimcodigo

Quando inicializamos o mouse, iremos tornar vazia nossa lista de
botões pressionados e soltos e também inicializamos como zero todos os
botões, bem como a posição, velocidade e aceleração.

\iniciocodigo
@<Inicializando ou Reinicializando o Mouse@>=
{
  int i;
  for(i = 0; i < 4; i ++){
    pressed_buttons[i].button = 0;
    released_buttons[i] = 0;
  }
  for(i = 0; i < W_MOUSE_SIZE; i ++)
    _Wmouse.button[i] = 0;
  _Wmouse.x = _Wmouse.y = _Wmouse.dx = _Wmouse.dy = _Wmouse.ddx = _Wmouse.ddy = 0;
  last_mouse_dx = last_mouse_dy = 0;
  mouse_initialization = 3;
  @<Obtém Posição do Mouse@>
}
@
\fimcodigo

\subsubsecao{3.3.1. Lendo o Mouse no X}

No X11, o número de botões suportados em um mouse é 5, e eles são
numerados entre 1 e 5. Iremos associar ao número zero o significado de
``qualquer botão foi pressionado''. Para um mouse, nós não precisamos
nos preocupar com a existência de diferentes mapeamentos que dão
significados diferentes para cada botão. Assim, podemos definir o
número de botões existentes e suas respectivas posições no vetor de
botões por meio das seguintes macros:

\iniciocodigo
@<Define Macros@>+=
#if !defined(_WIN32) && !defined(__EMSCRIPTEN__)
#define W_MOUSE_SIZE 6
#define W_MOUSE_LEFT   Button1
#define W_MOUSE_MIDDLE Button2
#define W_MOUSE_RIGHT  Button3
#define W_MOUSE_X1     Button4
#define W_MOUSE_X2     Button5
#endif
@
\fimcodigo

Sobre o pressionar de botões, para detectá-los devemos detectar
eventos to tipo \monoespaco{ButtonPress}. E então ler neles a
informação de qual botão do mouse foi pressionado, adicionando-o para
nossa lista de botões pressionados:

\iniciocodigo
@<X11: Tratar Eventos@>+=
if(event.type == ButtonPress){
  unsigned button = event.xbutton.button;
  @<Adiciona 'button' à Lista de Botões Pressionados@>
}
@
\fimcodigo

Para detectar um botão sendo solto, devemos checar por eventos do
tipo \monoespaco{ButtonRelease} e ler no evento qual dos possíveis
cinco botões foram soltos. Em seguida, nós removemos tal botão da
lista de botões pressionados e o colocamos na lista de botões soltos:

\iniciocodigo
@<X11: Tratar Eventos@>+=
if(event.type == ButtonRelease){
  unsigned button = event.xbutton.button;
  @<Remove 'botton' da Lista de Botões Pressionados e Adiciona à Botões Soltos@>
}
@
\fimcodigo

E também temos que detectar movimentos do mouse. Para isso devemos
esperar por eventos de movimento do mouse:

\iniciocodigo
@<X11: Tratar Eventos@>+=
if(event.type == MotionNotify){
  int x, y;
  x = event.xmotion.x;
  y = (window_size_y - 1) - event.xmotion.y;
  @<Atualiza Posição do Mouse 'x' e 'y'@>
}
@
\fimcodigo

Note que temos que fazer uma transformação na coordenada $y$, pois o
Xlib considera a posição zero o topo da janela, enquanto aqui tratamos
como zero a parte inferior da janela.

Além disso, caso o mouse nunca tenha sido movido, nós ainda temos que
saber qual a sua posição inicial. Para estes casos, usamos o código
abaixo que chama a função \monoespaco{XQueryPointer}. Esta função
retorna uma grande quantidade de informação sobre a posição do mouse
relativo não so à nossa janela atual, mas também relacionado à janela
acima e à janelas abaixo na hierarquia. Nós ignoramos a maioria de
tais informações, exceto pela posição relativa à nossa janela atual.

\iniciocodigo
@<Obtém Posição do Mouse@>=
#if !defined(_WIN32) && !defined(__EMSCRIPTEN__)
{
  int x, y;
  Window root_return, child_return;
  int root_x_return, root_y_return;
  unsigned mask_return;
  XQueryPointer(display, window, &root_return, &child_return,
                &root_x_return, &root_y_return, &x, &y, &mask_return);
  // Transformando coordenada 'y':
  y = (window_size_y - 1) - y;
  @<Atualiza Posição do Mouse 'x' e 'y'@>
}
#endif
@
\fimcodigo

\subsubsecao{3.3.2. Lendo o Mouse no Web Assembly}

Usando a API SDL fornecida pelo Emscripten, também suportamos um total
de cinco botões no Web Assembly, cada um deles numerado entre 1 e
5. Também podemos aqui usar a posição zero para representar ``qualquer
tecla'':

\iniciocodigo
@<Define Macros@>+=
#if defined(__EMSCRIPTEN__)
#define W_MOUSE_SIZE 6
#define W_MOUSE_LEFT   SDL_BUTTON_LEFT
#define W_MOUSE_MIDDLE SDL_BUTTON_MIDDLE
#define W_MOUSE_RIGHT  SDL_BUTTON_RIGHT
#define W_MOUSE_X1     SDL_BUTTON_X1
#define W_MOUSE_X2     SDL_BUTTON_X2
#endif
@
\fimcodigo

Detectamos que um botão do mouse foi pressionado com o código abaixo:

\iniciocodigo
@<Web Assembly: Tratar Eventos@>+=
if(event.type == SDL_MOUSEBUTTONDOWN){
  unsigned button = event.button.button;
  @<Adiciona 'button' à Lista de Botões Pressionados@>
}
@
\fimcodigo

Detectamos que o botão foi solto com o código abaixo:

\iniciocodigo
@<Web Assembly: Tratar Eventos@>+=
if(event.type == SDL_MOUSEBUTTONUP){
  unsigned button = event.button.button;
  @<Remove 'botton' da Lista de Botões Pressionados e Adiciona à Botões Soltos@>
}
@
\fimcodigo

E detectamos o movimento do mouse abaixo:

\iniciocodigo
@<Web Assembly: Tratar Eventos@>+=
if(event.type == SDL_MOUSEMOTION){
  int x, y;
  x = event.motion.x;
  y = (window_size_y - 1) - event.motion.y;
  @<Atualiza Posição do Mouse 'x' e 'y'@>
}
@
\fimcodigo

Já para obter a posição inicial do mouse, esmo que ele não tenhasido
movido, usamos o código abaixo:

\iniciocodigo
@<Obtém Posição do Mouse@>+=
#if defined(__EMSCRIPTEN__)
{
  int x, y;
  SDL_GetMouseState(&x, &y);
  // Transformando coordenada 'y':
  y = (window_size_y - 1) - y;
  @<Atualiza Posição do Mouse 'x' e 'y'@>
}
#endif
@
\fimcodigo

\subsubsecao{3.3.3. Lendo o Mouse no Windows}

No Windows nós não tems números que por padrão são associados a cada
um dos diferentes botões do mouse. Ao invés disso, cada botão gera um
evento diferente ao invés de gerar um evento de botão pressionado
genérico. Mas nós ainda precisamos associar cada botão a um número que
será sua posição no vetor de botões de nossa estrutura de mouse. Então
nós criamos nossa própria associação entre botões e números com as
macros abaixo:

\iniciocodigo
@<Define Macros@>+=
#if defined(_WIN32)
#define W_MOUSE_SIZE 6
#define W_MOUSE_LEFT   1
#define W_MOUSE_MIDDLE 2
#define W_MOUSE_RIGHT  3
#define W_MOUSE_X1     4
#define W_MOUSE_X2     5
#endif
@
\fimcodigo

Como o Windows cria eventos de tipos diferentes para cada um dos
botões, e além disso trata como se fôsse um único botão os botões X1 e
X2, apenas diferenciando-os por meio de um dos bits da mensagem, temos
trabalho adicional para tratar os eventos de clique de mouse:

\iniciocodigo
@<Windows: Tratar Eventos de Mouse@>=
if(event.message == WM_LBUTTONDOWN){
  unsigned button = W_MOUSE_LEFT;
  @<Adiciona 'button' à Lista de Botões Pressionados@>
}
else if(event.message == WM_MBUTTONDOWN){
  unsigned button = W_MOUSE_MIDDLE;
  @<Adiciona 'button' à Lista de Botões Pressionados@>
}
else if(event.message == WM_RBUTTONDOWN){
  unsigned button = W_MOUSE_RIGHT;
  @<Adiciona 'button' à Lista de Botões Pressionados@>
}
else if(event.message == WM_XBUTTONDOWN){
  unsigned button = W_MOUSE_X2;
  if((event.wParam >> 16) & 0x0001){
    unsigned button = W_MOUSE_X1;
  }
  @<Adiciona 'button' à Lista de Botões Pressionados@>
}
@
\fimcodigo

Para detectar que um dos botões clicados foi solto, também temos que
verificar um evento diferente para cada botão tratando os botões X1 e
X2 de maneira diferente dos demais:

\iniciocodigo
@<Windows: Manage Mouse Events@>+=
if(event.message == WM_LBUTTONUP){
  unsigned button = W_MOUSE_LEFT;
  @<Remove 'botton' da Lista de Botões Pressionados e Adiciona à Botões Soltos@>
}
else if(event.message == WM_MBUTTONUP){
  unsigned button = W_MOUSE_MIDDLE;
  @<Remove 'botton' da Lista de Botões Pressionados e Adiciona à Botões Soltos@>
}
else if(event.message == WM_RBUTTONUP){
  unsigned button = W_MOUSE_RIGHT;
  @<Remove 'botton' da Lista de Botões Pressionados e Adiciona à Botões Soltos@>
}
else if(event.message == WM_XBUTTONUP){
  unsigned button = W_MOUSE_X2;
  if((event.wParam >> 16) & 0x0001){
    unsigned button = W_MOUSE_X1;
  }
  @<Remove 'botton' da Lista de Botões Pressionados e Adiciona à Botões Soltos@>
}
@
\fimcodigo

E assim é como detectamos o evento no qual o mmouse se move:

\iniciocodigo
@<Windows: Manage Mouse Events@>+=
if(event.type == WM_MOUSEMOVE){
  int x, y;
  x = (event.lParam & 0xffff);
  y = (window_size_y - 1) - (event.lParam >> 16);
  @<Atualiza Posição do Mouse 'x' e 'y'@>
}
@
\fimcodigo

Como em todos os outros ambientes,mudamos a coordenada para que ela
fique relatia ao canto inferior esquerdo da tela, não o canto superior
esquerdo como é mais comum.

E para obter a posição inicial do mouse, mesmo que ele nunca tenha se
movido, usamos o código:

\iniciocodigo
@<Obtém Posição do Mouse@>+=
#if defined(_WIN32)
{
  int x, y;
  POINT point;
  // Obtém coordenadas em relação à tela:
  GetCursorPos(&point);
  // Converte para coordenada em relação à janela
  ScreenToClient(window, &point);
  // Armazena a coordenada, transformando a coordenada 'y':
  x = point.x;
  y = (window_size_y - 1) - point.y;
  @<Atualiza Posição do Mouse 'x' e 'y'@>
}
#endif
@
\fimcodigo


\subsecao{3.3.4. Código Adicional para Suporte de Mouse}

Depois de definirmos nas Subseções anteriores o código específico para
cada ambiente para detectarmos cliques e movimentos do mouse, agora
iremos lidar com o código comum a todos os ambientes.

Primeiro, iremos explicitar o código que adiciona um botão do mouse
pressionado à nossa lista de botões pressionados. Lembre-se que a
lista é um vetor de até 4 elementos onde marcamos qualquer fim
prematuro da lista com um botão nulo:

\iniciocodigo
@<Adiciona 'button' à Lista de Botões Pressionados@>=
{
  int i;
  for(i = 0; i < 4; i ++){
    if(pressed_buttons[i].button == button) // Já tava pressionado
      break;
    if(pressed_buttons[i].button == 0){ // Começou a ser pressionado agora
      pressed_buttons[i].button = button;
      pressed_buttons[i].time = current_time;
      // Atualizando mouse:
      _Wmouse.button[button] = 1;
      break;
    }
  }
  if(i == 4) continue; // Ignorando: muitos botões pressionados
}
@
\fimcodigo

Assim como podemos adicionar botões à esta lista, vamos também
removê-los com o código abaixo. E ao remover um botão da lista de
botões pressionados, o adicionamos necessariamente à lista de botões
soltos:

\iniciocodigo
@<Remove 'botton' da Lista de Botões Pressionados e Adiciona à Botões Soltos@>=
{
  int i;
  long stored_time = -1;
  for(i = 0; i < 4; i ++){ // Removendo da lsta de botões pressionados
    if(pressed_buttons[i].button == button){
      int j;
      stored_time = pressed_buttons[i].time;
      for(j = i; j < 3; j ++){
        pressed_buttons[j].button = pressed_buttons[j + 1].button;
        pressed_buttons[j].time = pressed_buttons[j + 1].time;
        if(pressed_buttons[j].button == 0)
          break;
      }
      pressed_buttons[3].button = 0;
      break;
    }
  }
  for(i = 0; i < 4; i ++){ // Adicionando à lista de botões soltos:
    if(released_buttons[i] == 0)
      released_buttons[i] = button;
  }
  if(i == 4) // Muitos botões soltos, ignore e limpe o mouse:
    _Wmouse.button[button] = 0;
  else{
    // Atualizando estrutura do mouse:
    _Wmouse.button[button] = - (long) (current_time - stored_time);
    if(_Wmouse.button[button] == 0)
      _Wmouse.button[button] = -1; //Apertou/soltou no mesmo frame
  }
}
@
\fimcodigo

Assim como tivemos que fazer no teclado, depois de atualizarmos a
lista de botões pressionados, temos que atualizar o vetor de botões do
mouse, atualizando informação de por quanto tempo cada um está sendo
pressionado:

\iniciocodigo
@<Depois de Obter Eventos da Janela@>+=
{
  int i;
  // Atualiza se existe qualquer botão sendo pressionado:
  _Wmouse.button[W_ANY] = (pressed_buttons[0].button != 0);
  for(i = 0; i < 4; i ++){
    if(pressed_buttons[i].button == 0)
      break;
    if(current_time > pressed_buttons[i].time)
      _Wmouse.button[pressed_buttons[i].button] =
        (current_time - pressed_buttons[i].button);
  }
}
@
\fimcodigo

Assim como antes de atualizarmos a lista de botões pressionados, temos
que esvaziar a lista de botões soltos e marcamos no mouse que aquele
boão não está mais sendo pressionado:

\iniciocodigo
@<Antes de Obter Eventos da Janela@>=
{
  int i;
  for(i = 0; i < 4; i ++){
    if(released_buttons[i] == 0)
      break;
    _Wmouse.button[released_buttons[i]] = 0;
    released_buttons[i] = 0;
  }
}
@
\fimcodigo

E agora mostramos o código para atualizar a posição do mouse, bem como
sua velocidade e aceleração (a depender do quanto avançamos na
inicialização do mouse). Nós rodamos este código ao detectar o
movimento do mouse e também quando lemos pela primeira vez a posição
inicial do mouse.

\iniciocodigo
@<Atualiza Posição do Mouse 'x' e 'y'@>=
{
  if(mouse_initialization < 3){ // Se sabemos a posição prévia
    _Wmouse.dx = (x - _Wmouse.x);
    _Wmouse.dy = (y - _Wmouse.y);
  }
  _Wmouse.x = x;
  _Wmouse.y = y;
  if(mouse_initialization < 2){ // Se sabemos a velocidade prévia
    _Wmouse.ddx = (_Wmouse.dx - last_mouse_dx);
    _Wmouse.ddy = (_Wmouse.dy - last_mouse_dy);
  }
  last_mouse_dx = _Wmouse.dx;
  last_mouse_dy = _Wmouse.dy;
  if(mouse_initialization > 0)
    mouse_initialization --;
}
@
\fimcodigo

\subsecao{3.4. Funções Adicionais de Entrada}

Tanto quando vamos ler a nossa entrada pela primeira vez, como quando
algo nos interrompe e ficamos um tempo sem atender a novos eventos de
entrada, uma grande quantidade de eventos de movimento de mouse e
teclas pressionadas pode se acumular e vir ao mesmo tempo. Nestes
casos, o melhor a fazer é só ignorar esse fluxo inicial, já que não
seremos capazes de tratá-lo corretamente. É para isso que usamos a
seguinte função:

\iniciocodigo
@<Funções da API@>+=
void _Wflush_window_input(void){
  // Lemos todos os eventos acumulados:
  _Wget_window_input(~0x0);
  // Esvaziamos tudo que temos sobre o teclado:
  @<Iniciando ou Reiniciando Teclado@>
  // E fazemos o mesmo para o mouse
  @<Inicializando ou Reinicializando o Mouse@>
}
@
\fimcodigo


\secao{X. Estrutura Final do Arquivo}

O arquivo com o código-fonte de todas as funções definidas neste
artigo terá a seguinte forma:

\iniciocodigo
@(src/window.c@>=
#include "window.h"
@<Cabeçalhos@>
@<Funções Locais@>
@<Variáveis Globais@>
@<Variáveis Locais@>
@<Funções da API@>
@
\fimcodigo



\fim
